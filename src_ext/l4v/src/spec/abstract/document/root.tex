%
% Copyright 2014, General Dynamics C4 Systems
%
% SPDX-License-Identifier: GPL-2.0-only
%

% LaTeX master document for abstract spec.

\documentclass[10pt,a4paper]{scrbook}

% These old font commands have been removed from newer versions of
% the scrbook document class, but isabelle.sty still uses them.
\DeclareOldFontCommand{\rm}{\normalfont\rmfamily}{\mathrm}
\DeclareOldFontCommand{\sf}{\normalfont\sffamily}{\mathsf}
\DeclareOldFontCommand{\tt}{\normalfont\ttfamily}{\mathtt}
\DeclareOldFontCommand{\bf}{\normalfont\bfseries}{\mathbf}
\DeclareOldFontCommand{\it}{\normalfont\itshape}{\mathit}
\DeclareOldFontCommand{\sl}{\normalfont\slshape}{\@nomath\sl}
\DeclareOldFontCommand{\sc}{\normalfont\scshape}{\@nomath\sc}

\usepackage{isabelle,isabellesym}

% further packages required for unusual symbols (see also isabellesym.sty)
% use only when needed
\usepackage{amssymb}                   % for \<leadsto>, \<box>, \<diamond>,
                                       % \<sqsupset>, \<mho>, \<Join>,
                                       % \<lhd>, \<lesssim>, \<greatersim>,
                                       % \<lessapprox>, \<greaterapprox>,
                                       % \<triangleq>, \<yen>, \<lozenge>
\usepackage[english]{babel}            % greek for \<euro>,
                                       % english for \<guillemotleft>,
                                       %             \<guillemotright>
                                       % default language = last
%\usepackage[latin1]{inputenc}         % for \<onesuperior>, \<onequarter>,
                                       % \<twosuperior>, \<onehalf>,
                                       % \<threesuperior>, \<threequarters>,
                                       % \<degree>
%\usepackage[only,bigsqcap]{stmaryrd}  % for \<Sqinter>
%\usepackage{eufrak}                   % for \<AA> ... \<ZZ>, \<aa> ... \<zz>
                                       % (only needed if amssymb not used)

\usepackage{graphicx}
\usepackage[draft]{fixme}
\usepackage{cite}
\usepackage{xspace}

\usepackage{color}
\definecolor{linkcolor}{rgb}{0,0,0.7}
% this should be the last package used
\usepackage[colorlinks=true,linkcolor=linkcolor,citecolor=linkcolor,
            filecolor=linkcolor,pagecolor=linkcolor,urlcolor=linkcolor]{hyperref}

\urlstyle{rm}
\isabellestyle{tt}

\newcommand{\version}{\input{VERSION}\xspace}
\newcommand{\arch}{X64
\xspace}

\hypersetup
{
    pdfauthor={Trustworthy Systems, Data61},
    pdftitle={Abstract Formal Specification of the seL4/\arch API}
}

\renewcommand{\isamarkupchapter}[1]{\chapter{#1}}

% don't show ML code
\isadroptag{ML}

% don't show proofs
\isadroptag{proof}

% do not show confusing double quotes in definitions and lemmas
%\renewcommand{\isachardoublequote}{}
%\renewcommand{\isachardoublequoteopen}{}
%\renewcommand{\isachardoublequoteclose}{}

\parindent 0pt\parskip 0.5ex

\setlength{\oddsidemargin}{0cm}
\setlength{\evensidemargin}{0cm}
\setlength{\topmargin}{-1cm}
\setlength{\textwidth}{15.5cm}
\setlength{\textheight}{22.5cm}

\newcommand{\meth}[1]{\texttt{#1()}}
\newcommand{\obj}[1]{\textsf{\small #1}}

\begin{document}

% tex hacking to get git revision from repo document

% we can't include gitrev.tex in document_files, because this will rebuild
% ASpec (and therefore pretty much everything) every time the git revision
% changes (for any change anywhere). This means gitrev.tex will not be
% copied into the document build directory, and we have to find the repo
% checkout. We give the root of the repo checkout as document file, because
% if that changes, everything changes anyway. From there we can find the rest,
% which we do below:

\newcommand\gitroot{}
\newcommand\gitrev{}

\makeatletter\let\myfilehandle\@inputcheck\makeatother
\openin\myfilehandle=git-root.tex\relax
\begingroup\endlinechar-1
  \global\read\myfilehandle to \gitroot
\endgroup
\closein\myfilehandle

\makeatletter\let\myfilehandle\@inputcheck\makeatother
\openin\myfilehandle=\gitroot/spec/abstract/document/gitrev.tex\relax
\begingroup\endlinechar-1
  \global\read\myfilehandle to \gitrev
\endgroup
\closein\myfilehandle

% end tex hacking


\title{Abstract Formal Specification of the seL4/\arch API}

\date{Version \version}

\author{%
June Andronick \and
Joel Beeren \and
Matthew Brecknell \and
Timothy Bourke \and
Philip Derrin \and
Kevin Elphinstone \and
Xin Gao \and
David Greenaway \and
Alejandro Gomez-Londono \and
Gerwin Klein \and
Rafal Kolanski \and
Ramana Kumar \and
Daniel Matichuk \and
Thomas Sewell \and
Michael Sproul \and
Miki Tanaka \and
Sophie Taylor \and
Simon Winwood
}

\maketitle

\thispagestyle{empty}

\vfill

\copyright~2014 National ICT Australia Limited.\\
\copyright~2014 Open Kernel Labs, Inc.\\
\copyright~2018 Data61, CSIRO.\\

\textsc{All rights reserved}.

\bigskip

Architecture: \arch\\
Document build date: \today\\
Produced from git change set: \gitrev

\clearpage

\chapter*{Abstract}
This document is the text version of the abstract, formal
Isabelle/HOL specification of the seL4 microkernel. It is
intended to give a precise, operational definition of the
seL4 microkernel on the \arch architecture.
The document contains a short overview, followed by
text generated from the formal Isabelle/HOL definitions.

This document is not a tutorial or user manual and is not intended to be read
as such. Please see the bundled user manual for a higher-level introduction to
the kernel.


\cleardoublepage

\tableofcontents

\input{session}

\bibliographystyle{plain}
\bibliography{defs,root}

\end{document}

%%% Local Variables:
%%% mode: latex
%%% TeX-master: t
%%% End:
