%
% Copyright 2020, Data61, CSIRO (ABN 41 687 119 230)
%
% SPDX-License-Identifier: CC-BY-SA-4.0
%

\documentclass[11pt,a4paper]{scrreprt}

% These old font commands have been removed from newer versions of
% the scrreprt document class, but isabelle.sty still uses them.
\DeclareOldFontCommand{\rm}{\normalfont\rmfamily}{\mathrm}
\DeclareOldFontCommand{\sf}{\normalfont\sffamily}{\mathsf}
\DeclareOldFontCommand{\tt}{\normalfont\ttfamily}{\mathtt}
\DeclareOldFontCommand{\bf}{\normalfont\bfseries}{\mathbf}
\DeclareOldFontCommand{\it}{\normalfont\itshape}{\mathit}
\DeclareOldFontCommand{\sl}{\normalfont\slshape}{\@nomath\sl}
\DeclareOldFontCommand{\sc}{\normalfont\scshape}{\@nomath\sc}

\newif \ifDraft         \Draftfalse

\usepackage{isabelle,isabellesym}

% further packages required for unusual symbols (see also
% isabellesym.sty), use only when needed

%\usepackage{amssymb}
  %for \<leadsto>, \<box>, \<diamond>, \<sqsupset>, \<mho>, \<Join>,
  %\<lhd>, \<lesssim>, \<greatersim>, \<lessapprox>, \<greaterapprox>,
  %\<triangleq>, \<yen>, \<lozenge>

\usepackage[english]{babel}
  %option greek for \<euro>
  %option english (default language) for \<guillemotleft>, \<guillemotright>

%\usepackage[only,bigsqcap]{stmaryrd}
  %for \<Sqinter>

%\usepackage{eufrak}
  %for \<AA> ... \<ZZ>, \<aa> ... \<zz> (also included in amssymb)

%\usepackage{textcomp}
  %for \<onequarter>, \<onehalf>, \<threequarters>, \<degree>, \<cent>,
  %\<currency>


% Extra CAmkES bits.
\usepackage{graphicx}
\usepackage{enumerate}

% From ERTOS setup
\setkeys{Gin}{keepaspectratio=true,clip=true,draft=false,width=\linewidth}
\usepackage{times,cite,url,fancyhdr,microtype,color,geometry}

\renewcommand{\ttdefault}{cmtt}        % CM rather than courier for \tt

\usepackage{xspace}
\usepackage{listings}
\newcommand{\camkeslisting}[1]{{\lstset{basicstyle=\small\ttfamily,keywordstyle=\bf,morekeywords={assembly,component,composition,connection,control,consumes,dataport,emits,event,from,in,inout,int,out,procedure,provides,smallstring,to,uses}}\lstinputlisting{#1}}}


\ifDraft
\usepackage{draftcopy}
\newcommand{\Comment}[1]{\textbf{\textsl{#1}}}
\newcommand{\FIXME}[1]{\textbf{\textsl{FIXME: #1}}}
\newcommand{\todo}[1]{\textbf{TODO: \textsl{#1}}}
\date{\small\today}
\else
\newcommand{\Comment}[1]{\relax}
\newcommand{\FIXME}[1]{\relax}
\newcommand{\todo}[1]{\relax}
\date{}
\fi


% From camkes manual
\newcommand{\selfour}{seL4\xspace}
\newcommand{\Selfour}{SeL4\xspace}
\newcommand{\camkes}{CAmkES\xspace}

\newcommand{\code}[1]{\texttt{#1}}


\newcommand{\titl}{CAmkES Glue Code Semantics}
\newcommand{\authors}{Matthew Fernandez, Peter Gammie, June Andronick, Gerwin Klein, Ihor Kuz}

\definecolor{lcol}{rgb}{0,0,0.5}
\usepackage[bookmarks,hyperindex,pdftex,
            colorlinks=true,linkcolor=lcol,citecolor=lcol,
            filecolor=lcol,urlcolor=lcol,
            pdfauthor={\authors},
            pdftitle={\titl},
            plainpages=false]{hyperref}


\addto\extrasenglish{%
\renewcommand{\chapterautorefname}{Chapter}
\renewcommand{\sectionautorefname}{Section}
\renewcommand{\subsectionautorefname}{Section}
\renewcommand{\subsubsectionautorefname}{Section}
\renewcommand{\appendixautorefname}{Appendix}
\renewcommand{\Hfootnoteautorefname}{Footnote}
}

% urls in roman style
\urlstyle{rm}

\lstset{basicstyle=\small\tt}

% isabelle style
\isabellestyle{tt}

% for uniform isabelle font size
\renewcommand{\isastyle}{\isastyleminor}

% Abstract various things that might change.
\newcommand{\ccode}[1]{\texttt{#1}}
\newcommand{\isabelletype}[1]{\emph{#1}}
\newcommand{\isabelleterm}[1]{\emph{#1}}

\newcommand{\nictafundingacknowledgement}{%
% Lifted from http://wiki.inside.nicta.com.au/display/CHNOPSR/Funding+Acknowledgement 28-10-2013
NICTA is funded by the Australian Government through the Department of Communications and the Australian Research Council through the ICT Centre of Excellence Program. NICTA is also funded and supported by the Australian Capital Territory, the New South Wales, Queensland and Victorian Governments, the Australian National University, the University of New South Wales, the University of Melbourne, the University of Queensland, the University of Sydney, Griffith University, Queensland University of Technology, Monash University and other university partners.}

\newcommand{\ABN}{ABN 62 102 206 173}

\newcommand{\cpright}{Copyright \copyright\ 2013 NICTA, \ABN.  All rights reserved.}
\newcommand{\disclaimer}{%
\cpright}

\newcommand{\trdisclaimer}{%
This material is based on research sponsored by Air Force Research Laboratory
and the Defense Advanced Research Projects Agency (DARPA) under agreement number
FA8750-12-9-0179. The U.S. Government is authorized to reproduce and distribute
reprints for Governmental purposes notwithstanding any copyright notation
thereon.

The views and conclusions contained herein are those of the authors and should
not be interpreted as necessarily representing the official policies or
endorsements, either expressed or implied, of Air Force Research Laboratory,
the Defense Advanced Research Projects Agency or the U.S.Government.}

\newcommand{\smalldisclaimer}{}
\newcommand{\bigdisclaimer}{%
\nictafundingacknowledgement\\

\cpright\\

\vspace{2ex}
\noindent\trdisclaimer}

\newcommand{\pgstyle}{%
\fancyhf{}%
\renewcommand{\headrulewidth}{0pt}%
\fancyfoot[C]{}%
\fancyfoot[L]{\smalldisclaimer}%
\fancyfoot[R]{\sl\thepage}}

\fancypagestyle{plain}{\pgstyle}


\begin{document}

\parindent 0pt\parskip 0.5ex plus 0.2ex minus 0.1ex

%--------- title page
\newgeometry{left=25mm,right=25mm,top=35mm,bottom=35mm}

\vspace{14ex}
\textsf{\huge \titl}

%\vspace{2ex}
%\textsf{\huge \subtitl}

\vspace{4ex}
\rule{0.85\textwidth}{5pt}
\vspace{4ex}

{\large \authors

\vspace{2ex}
April 2013}

\vfill
{\small
\bigdisclaimer
}

\thispagestyle{empty}
\newpage
~
\restoregeometry

\fancypagestyle{empty}{\pgstyle}
\pagestyle{empty}

%--------- end title page

\cleardoublepage

\chapter*{Abstract}

This document describes the formal dynamic semantics of \camkes glue code, in
particular of the communication stubs generated for components at compile
time. The semantics is based on a simple concurrent imperative language with
message passing that is easy to extend and instantiate for specific
applications. Instead of one generic semantics for all systems, we take the
approach of generating a high-level semantic description for each specific
ADL component specification to ease verification of specific systems in
the future.

We show the definitions and types for expressing components and glue code, and
provide some examples of generated Isabelle theories with synchronous,
asynchronous, and shared memory communication.


\cleardoublepage
\tableofcontents

%
% Copyright 2014, NICTA
%
% This software may be distributed and modified according to the terms of
% the GNU General Public License version 2. Note that NO WARRANTY is provided.
% See "LICENSE_GPLv2.txt" for details.
%
% @TAG(NICTA_GPL)
%

\chapter{Introduction}

This document provides formal proofs in the interactive theorem prover
Isabelle/HOL~\cite{Nipkow_PW:Isabelle} of certain correctness properties of
generated communication code of the \camkes platform~\cite{Kuz_LGH_07}.
These proofs are example output of a generalised proof generation tool and
are intended to extend previous reports that describe the
static~\cite{Fernandez_KKM_13:tr} and dynamic~\cite{Fernandez_GAKK_13:tr}
semantics of \camkes systems.
The previous formalisms gave a high-level specification of the behaviour of
\camkes systems, while the current proofs reason about the behaviour of the
glue code at the level of C, targeting the seL4
microkernel~\cite{Klein_EHACDEEKNSTW_09}.

The proofs that follow are constructed on an abstraction of C code.
This abstraction process is performed by two existing
tools, a translation from C to the generic imperative language
SIMPL~\cite{Winwood_KSACN_09}, and a further abstraction by the tool
AutoCorres~\cite{Greenaway_LAK_14}, neither of which are specific
to \camkes.
These tools lift a C translation unit into monadic specifications of the
contained functions.
Alongside the generated code, we automate the production of proofs of
correctness properties of these specifications.
These proofs leverage an Isabelle
tactic that performs weakest pre-condition reasoning.

\camkes has three different communication modes: synchronous, asynchronous and
shared memory.
These are implemented as three \camkes primitives, remote procedure calls
(RPCs), events and dataports, respectively.
The desirable correctness property of connector glue code is dependent on which
of these the connector implements.
For example, remote procedure call connectors should ensure, among other
things, that the function call and parameters that are sent by the caller are
correctly received and decoded by the callee.
A common requirement for all the glue code is safe execution with respect to
the C standard and the state of the system at runtime.
The generated proofs given in the following chapters show this for three
specific connectors, one for each \camkes communication primitive, but the
proof generation process generalises to other \camkes connectors as well.
This property requires that the glue code only accesses valid memory, that it
obeys the restrictions of the C99 standard~\cite{C99} and that it always
terminates.

In proving this behaviour of the glue code, we rely on some explicit
assumptions on user code within the system.
In particular, we assume that the user code also obeys the C99 standard and
does not modify any glue code state.
The glue code state covers memory regions relevant for communication with seL4,
thread identification and thread-local storage.
This state is disjoint from the expected user state; that is, non-malicious
user code should never have cause to modify any of the glue code state.
As for the seL4 proofs, the generated proofs of CAmkES glue code are intended
to apply to an ARM, unicore platform and may not hold in other operating
environments.

The connectors on which the generated proofs below are based have some
limitations that we make explicit here.
The event connector used does not support callbacks --
events can only be waited on or polled for.
The RPC connector used does not support array parameters, strings or
user-defined data types (e.g. C structs).
These are limitations that will be lifted in future.


% generated text of all theories
\input{session}

% optional bibliography
\bibliographystyle{plain}
\bibliography{root}

\end{document}
