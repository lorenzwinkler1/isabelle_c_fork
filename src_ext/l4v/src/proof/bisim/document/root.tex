%
% Copyright 2020, Data61, CSIRO (ABN 41 687 119 230)
%
% SPDX-License-Identifier: CC-BY-SA-4.0
%

% LaTeX master document for abstract spec.

\documentclass[10pt,a4paper]{scrbook}

% These old font commands have been removed from newer versions of
% the scrbook document class, but isabelle.sty still uses them.
\DeclareOldFontCommand{\rm}{\normalfont\rmfamily}{\mathrm}
\DeclareOldFontCommand{\sf}{\normalfont\sffamily}{\mathsf}
\DeclareOldFontCommand{\tt}{\normalfont\ttfamily}{\mathtt}
\DeclareOldFontCommand{\bf}{\normalfont\bfseries}{\mathbf}
\DeclareOldFontCommand{\it}{\normalfont\itshape}{\mathit}
\DeclareOldFontCommand{\sl}{\normalfont\slshape}{\@nomath\sl}
\DeclareOldFontCommand{\sc}{\normalfont\scshape}{\@nomath\sc}

\usepackage{isabelle,isabellesym}

% further packages required for unusual symbols (see also isabellesym.sty)
% use only when needed
\usepackage{amssymb}                   % for \<leadsto>, \<box>, \<diamond>,
                                       % \<sqsupset>, \<mho>, \<Join>,
                                       % \<lhd>, \<lesssim>, \<greatersim>,
                                       % \<lessapprox>, \<greaterapprox>,
                                       % \<triangleq>, \<yen>, \<lozenge>
\usepackage[english]{babel}            % greek for \<euro>,
                                       % english for \<guillemotleft>,
                                       %             \<guillemotright>
                                       % default language = last
%\usepackage[latin1]{inputenc}         % for \<onesuperior>, \<onequarter>,
                                       % \<twosuperior>, \<onehalf>,
                                       % \<threesuperior>, \<threequarters>,
                                       % \<degree>
%\usepackage[only,bigsqcap]{stmaryrd}  % for \<Sqinter>
%\usepackage{eufrak}                   % for \<AA> ... \<ZZ>, \<aa> ... \<zz>
                                       % (only needed if amssymb not used)

\usepackage{graphicx}
\usepackage[draft]{fixme}
\usepackage{cite}

% this should be the last package used
\usepackage{color}
\definecolor{linkcolor}{rgb}{0,0,0.7}
\usepackage[colorlinks=true,linkcolor=linkcolor,citecolor=linkcolor,
            filecolor=linkcolor,pagecolor=linkcolor,urlcolor=linkcolor]{hyperref}

\urlstyle{rm}
\isabellestyle{tt}

% FIXME: sseefried: change the pdftitle
\hypersetup
{
    pdfauthor={NICTA SSRG Research Group},
    pdftitle={Bisimlulation proof}
}

\renewcommand{\isamarkupheader}[1]{\chapter{#1}}

% don't show ML code
\isadroptag{ML}

% don't show proofs
\isadroptag{proof}

% do not show confusing double quotes in definitions and lemmas
%\renewcommand{\isachardoublequote}{}
%\renewcommand{\isachardoublequoteopen}{}
%\renewcommand{\isachardoublequoteclose}{}

\parindent 0pt\parskip 0.5ex

\setlength{\oddsidemargin}{0cm}
\setlength{\evensidemargin}{0cm}
\setlength{\topmargin}{-1cm}
\setlength{\textwidth}{15.5cm}
\setlength{\textheight}{22.5cm}

\newcommand{\meth}[1]{\texttt{#1()}}
\newcommand{\obj}[1]{\textsf{\small #1}}

\begin{document}

% FIXME: sseefried: Change the title
\title{Bisimulation Proof}

% FIXME: sseefried: Is this all the authors?
\author{%
Simon Winwood \and
Gerwin Klein \and
Sean Seefried \and
}


\maketitle

\thispagestyle{empty}

\vfill

\copyright~2013 National ICT Australia Limited.\\
% FIXME: sseefried: Should we have a copyright for Open Kernel Labs, Inc as well. Uncomment below if so.
% \copyright~2013 Open Kernel Labs, Inc.\\

% FIXME: sseefried: This license is almost certainly not correct. Change it.
\textsc{All rights reserved}. This document is made available under
the terms of the National ICT Australia Limited
\textsc{Non-Commercial License Agreement} and Open Kernel Labs,
Inc.\ \textsc{Non-Commercial License Agreement}. Copies of the
licenses are available in the top-level directory of this
distribution.

\clearpage

\chapter*{Abstract}
% FIXME: sseefried: Write an abstract for the document
%This document is the text version of the abstract, formal
%Isabelle/HOL specification of the seL4 microkernel. It is
%intended to give a precise, operational definition of the
%seL4 microkernel on the ARMv6 architecture.
%The document contains a short overview, followed by
%text generated from the formal Isabelle/HOL definitions.

This document is not a tutorial or user manual and is not intended to be read
as such. Please see the bundled user manual for a higher-level introduction to
the kernel.


\cleardoublepage

\tableofcontents

\input{session}

% \bibliographystyle{plain}
% \bibliography{defs,root}

\end{document}

%%% Local Variables:
%%% mode: latex
%%% TeX-master: t
%%% End:
