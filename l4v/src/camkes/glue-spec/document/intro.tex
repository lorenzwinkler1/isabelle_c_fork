%
% Copyright 2014, NICTA
%
% This software may be distributed and modified according to the terms of
% the GNU General Public License version 2. Note that NO WARRANTY is provided.
% See "LICENSE_GPLv2.txt" for details.
%
% @TAG(NICTA_GPL)
%

\chapter{Introduction}
\label{s:intro}

This document shows the formal Isabelle/HOL~\cite{Nipkow_PW:Isabelle}
specification of the behaviour of \camkes
component systems defined by ADL descriptions \cite{Kuz_LGH_07}.
This formal specification is intended to apply to the glue code; the generated
communication stubs that are provided to the user by the \camkes platform.
It extends and supplements the previous
report which described the static semantics of component and system
specifications \cite{Kuz_FKM_12}.
Together these two reports form the formal specification of the \camkes ADL.
Although this is its most interesting part, the specification presented here
goes beyond providing just the glue code semantics. Instead we provide an
abstract high-level specification of the behaviour of an entire \camkes
component system.

The specification is high-level, because it abstracts from the underlying
kernel mechanisms and message formats. Instead, it is based on a general
concurrent message passing framework that can transmit messages of arbitrary
high-level types. Instantiating this framework we restrict it to the kinds of
message types of the ADL description and map \camkes mechanisms to the
message passing primitives. Showing that the kernel and glue code indeed
implement this high-level semantic view is the main proof obligation of the
future glue code correctness proof.

The idea of the specification is to provide a high-level view of the behaviour
of a component system using semantic mechanisms that nevertheless map
reasonably easily to the glue code implementation and the underlying kernel
mechanisms that provide architecture and communication boundary enforcement.
The basic communication principle of the underlying semantic framework is
synchronous message passing. This is presented in a way that makes it convenient to
additionally model atomic asynchronous events and shared memory reads/writes by
adding intermediate simulated components. These intermediate model processes map to
kernel event buffers and the usual behaviour of shared memory
pages. At the expense of making the shared memory component more complex, it
would be feasible to explicitly include the effects of
weak memory models. We do not do this here, because the intended application
scenario is a unicore setting.

Similar to how the instantiation of a component system is generated from its
ADL description in conjunction with provided user code, we generate
the formal specification of a complete \camkes component system from the same
ADL description together with a set of generic base definitions in this
document and a set of user-provided behaviour definitions for trusted
components. Trusted components are those that are claimed to be more
constrained in their behaviour than the architecture boundaries enforce.

The remainder of this report is structured as follows. We first introduce the
semantic concurrency framework the glue code definitions are based on, in
\autoref{s:cimp}. We then
proceed to define the basic data types that instantiate this semantic
framework to \camkes systems, in \autoref{h:types}. After this, in
\autoref{h:abbreviations} we can define
the building blocks
that the generated glue code specifications will further instantiate and use.
\autoref{h:connector} defines the intermediate event and memory components
mentioned above and \autoref{h:userstubs} provides default instantiations of
types and definitions that the user may choose to override.

\autoref{h:procbase}, \autoref{s:eventbase} and \autoref{h:dataportbase}
show example specifications produced from a number of small \camkes ADL
descriptions, illustrating the output of the generation phase. These examples
show the actual glue code specifications that fit the corresponding generated
C code. A more detailed system is presented in \autoref{h:filter} to
illustrate how to define and use trusted components.
