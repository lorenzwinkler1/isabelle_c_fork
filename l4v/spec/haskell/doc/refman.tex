%
% Copyright 2014, General Dynamics C4 Systems
%
% This software may be distributed and modified according to the terms of
% the GNU General Public License version 2. Note that NO WARRANTY is provided.
% See "LICENSE_GPLv2.txt" for details.
%
% @TAG(GD_GPL)
%

\documentclass[a4paper,11pt,twoside]{report}
\usepackage[colour,nictaonly]{disy}

\usepackage{comment}
\usepackage{ifxetex,ifpdf}
\usepackage{alltt}
\usepackage[T1]{fontenc}

\usepackage[margin=33mm]{geometry}
\usepackage{cite,fancyhdr,microtype,tocloft,times}

\pagestyle{fancyplain}
\lhead[\fancyplain{}{\sl\thepage}]{\fancyplain{}{\sl\rightmark}}
\chead{}
\rhead[\fancyplain{}{\sl\leftmark}]{\fancyplain{}{\sl\thepage}}
\lfoot[\fancyplain{\sl\thepage}{}]{}
\cfoot{}
\rfoot[]{\fancyplain{\sl\thepage}{}}

\definecolor{linkcolor}{rgb}{0,0,0.4}
\usepackage[colorlinks=true,linkcolor=linkcolor,citecolor=linkcolor,
            filecolor=linkcolor,urlcolor=linkcolor]{hyperref}

\setlength{\cftsubsecnumwidth}{4em}
\setlength{\cftsecnumwidth}{6ex}

\newcommand{\softwareversion}{\input{version.txt}}

% Definitions of the appearance of the typeset Haskell code
\ifxetex
\usepackage{fontspec}
\defaultfontfeatures{Scale=MatchLowerCase,Mapping=tex-text}
%\newfontfamily\hsfamily[Scale=0.85,BoldFont={Latin Modern Mono}, SmallCapsFont={Latin Modern Mono Caps}]{Latin Modern Mono Light}
\newfontfamily\hsfamily[Scale=0.91,SmallCapsFont={Fontin Sans Small Caps}]{Fontin Sans}
\else
%\newcommand{\hsfamily}{\ttfamily}
\newcommand{\hsfamily}{\sffamily}
\fi

% Command used to mark Haskell-specific implementation details so we can easily remove them from the manual
%\includecomment{impdetails}
\excludecomment{impdetails}
%\newif \ifimpdetails \impdetailstrue
%\newenvironment{impdetails}{
% \ifimpdetails\begin{comment}\fi}{\ifimpdetails\end{comment}\fi}

\usepackage{lambdaTeX}

% Workaround for conflict between PDFTeX and lambdaTeX
\catcode`\<=12
\catcode`\>=12

% XXX work around broken tex-text mapping
\ifxetex\typeset{`}{\hbox{\hsfamily\addfontfeatures{Mapping=none} `}}\fi

% Symbols for bitwise AND and OR
\typeset{.|.}{$\ddot{\vee}$}
\typeset{.&.}{$\ddot{\wedge}$}

% Symbols for monad bind operators
\typeset{>>}{$\gg$}
\typeset{<<}{$\ll$}
\typeset{>>=}{$\gg\joinrel\joinrel\Relbar$}
\typeset{=<<}{$\Relbar\joinrel\joinrel\ll$}

\typeout{If TeX stops here, run mkfunctions.pl to generate functions.aux}
\input{functions.aux}

\hyphenation{C-Space C-Spaces C-Node C-Nodes Has-kell Send-Recv Ex-change-Reg-ist-ers Thread-Con-trol}

\begin{document}

\title{seL4 Haskell Manual}
\author{Trustworthy~Systems~Team, NICTA}
\AuthorEmail{ssrg@nicta.com.au}
\date{\softwareversion}

\maketitle

  \urlstyle{sf}
  \thispagestyle{empty}

  \vfill

  \copyright\ General Dynamics C4 Systems.\\

  \textsc{All rights reserved}.

  % Acknowledgements
  \thispagestyle{empty}
  \vfill
  \renewcommand{\abstractname}{Acknowledgements}
  \begin{abstract}
The seL4 project team would like to acknowledge the following people
and teams (in no particular order) who have contributed to seL4 in
some way. Apologies if we missed you.

seL4 is a descendant of L4 itself, and thus owes a lot to Jochen
Liedtke's work on microkernels.

The L4Ka team at the University of Karlsruhe and Hermann H\"{a}rtig's
operating system group at TU Dresden have provided many fruitful
discussions that have influenced seL4's design.

Jonathon Shapiro has also provided insightful interaction on many
occasions, and his work on EROS and its successor Coyotos has also
influenced seL4's design.

We would also like to thank Gerwin Klein and his L4.verified
project team at NICTA for their insights and continual feedback on
our work. David Cock deserves special mention for his work on
bringing the ARM simulator into being.

\vspace{5ex}
\begin{center}
{\large\textbf{Warning}}
\end{center}
This manual is currently out of date. While the Haskell code itself is kept
up-to-date, the text in this manual describes an old version of seL4 and uses
outdated/deprecated terminology.
\begin{center}
Use with caution.
\end{center}
\end{abstract}

  \thispagestyle{empty}

  \cleardoublepage
  \setcounter{page}{1}
  \tableofcontents

  \cleardoublepage
  \setcounter{page}{1}
  \pagenumbering{arabic}

% Re-enable lambdaTeX's active < and >
\catcode`\<=\active
\catcode`\>=\active


\part{Introduction}

\chapter{Introduction}

This manual documents the seL4 kernel reference implementation, which
is written in Haskell. It provides annotated Haskell source code
describing the high-level, architecture independent behaviour of the
three services that the kernel provides --- threads, address spaces,
and communication --- including the system calls used to control them
from user level. This source code is executable --- not pseudocode ---
and forms part of a simulator which can be used to examine the
behaviour of a running seL4 kernel.

\autoref{sec:overview} provides an overview of the seL4 API. \autoref{sec:haskell} briefly discusses Haskell, and the manner in which we have applied it to construct the reference implementation. The remaining chapters are generated directly from the Haskell source code of the reference implementation, and contain all of the code (apart from some Haskell-specific implementation details, such as compiler and pre-processor directives).

\chapter{seL4 Overview}\label{sec:overview}
%
% Copyright 2014, General Dynamics C4 Systems
%
% This software may be distributed and modified according to the terms of
% the GNU General Public License version 2. Note that NO WARRANTY is provided.
% See "LICENSE_GPLv2.txt" for details.
%
% @TAG(GD_GPL)
%

\section{Introduction}\label{sec:overview.intro}

This chapter provides an overview of the \emph{seL4} kernel. It
concentrates on the conceptual differences between seL4 and existing
L4 specifications.

The seL4 kernel's design focuses on two general issues with the
existing L4 API: authorisation of system calls and kernel resource
management. seL4 addresses these general areas by using capabilities
to authorise system calls and making kernel data structures first
class objects whose allocation is performed by user-level operating
system personalities.

\section{Capabilities, Endpoints, and System Calls}\label{sec:overview.caps}

Each user-space thread in the system possesses a set of \emph{capabilities}. A
capability provides the thread that possesses it with the ability to perform one
or more kernel operations. A thread may address its capabilities using their
locations in that thread's \emph{capability address space} (henceforth
abbreviated \emph{CSpace}).

A capability contains a reference to an area of memory that is used to support
the kernel services that the capability provides. It also contains data
identifying the type of services provided. Capabilities are immutable and
opaque from the point of view of the user threads possessing them; once a
capability has been created, its contents are only visible to the kernel.

\subsection{Capability Invocation}

Capabilities may be \emph{invoked} by passing their CSpace addresses to one of
the kernel's two basic operations --- \emph{Send} and \emph{Recv}. The former
is used to send a message to the object represented by the capability, and the
latter to wait for a reply.

The potential recipients or senders of a message through an invoked capability
depend on the type of the kernel object that the capability refers to. For
most object types, the kernel itself is the recipient of the message; however,
messages may also be sent to other user-level threads.

\subsection{Synchronous Endpoints and IPC}\label{sec:overview.caps.ipc}

The basic inter-process communication (or \emph{IPC}) operation is coordinated
using \emph{endpoint} objects. An endpoint is a small kernel object which
contains a queue of threads, and a state flag that indicates whether the
threads in the queue are waiting for Send operations or Recv operations. It
acts as a one-way channel for communication between one or more senders and
one or more recipients. Unlike previous versions of L4, the global identities
of the participants in IPC are not exposed; threads send and receive IPC using
only the local addresses of their endpoints.

Multiple threads may perform Send or Recv operations on the same endpoint.
When a pair of threads invoke an endpoint, performing a Send operation and a
Recv operation respectively, the kernel will perform a \emph{message transfer}
between them. Threads which cannot immediately be paired are suspended until a
partner arrives. If multiple threads are suspended on one endpoint, the
kernel's algorithm for choosing a thread to resume is not specified, but
typically will be first in, first out order.

\subsubsection{Badged Messages}

\begin{figure}
\centering \includegraphics{figures/ipctransfer}
\caption{IPC operation, without message buffers. The solid lines represent
references to kernel objects; the dashed lines represent the transfer of
control and data between the threads.}
\label{fig:ipctransfer}
\end{figure}

Since it is only possible to wait for messages from one specific endpoint at a
time, a server wishing to provide multiple interfaces or to service requests
from multiple clients must provide capabilities to the same endpoint for every
interface or client. The kernel provides the server with a means to identify
the capability used to send a message (and therefore the client or interface):
\emph{badged} endpoint capabilities.

Threads holding an endpoint capability with the right to perform the
Mint operation on it may create new capabilities to that endpoint which
are marked with badges. A badge is a data word with a meaning defined by
the user-level server that creates it; typically it will be an index into a
table of interfaces provided by the server. The badge on a capability is
immutable and opaque: a badged capability cannot be used with the Mint operation, and a badge cannot be inspected by any means other than receiving a message that was sent with it.

When a message is sent using a badged endpoint capability, the badge is passed
to the recipient along with the rest of the message. See
\autoref{fig:ipctransfer} for an example.

\subsubsection{Message Transfers}

\begin{figure}
\centering \includegraphics{figures/truncation}
\caption{Truncation of a message when one of the IPC buffer capabilities is
missing.}
\label{fig:truncation}
\end{figure}

Message transfers work by transferring the contents of a certain number of
\emph{message words} from the sender to the recipient. A small number of
these will be transferred in the CPU's general purpose registers --- the number
of registers used for this purpose depends on the architecture. If there is
more data to be transferred, then per-thread \emph{message buffers} are used.

The first word of the message contains parameters that affect the transfer of
the message, including the number of words after the first that will be
transferred. If the kernel is unable to perform the transfer as requested, the
recipient (but \emph{not} the sender) will receive a modified version of the
first word that reflects the actual transfer performed.

If the message buffer must be used to transfer the requested number of message
words, and either the
sender's or the recipient's message buffer capability is missing, the message
will be truncated as shown in \autoref{fig:truncation}. Note that the
sender cannot tell that this has happened, as informing it would open a covert
channel from the recipient to the sender. The recipient is not explicitly
notified either, though the message length specified in the first word will be
reduced to reflect the amount of data actually sent. The recipient may then use
protocol-specific means to detect truncation.

It is the responsibility of each of the two
threads involved, and their pagers, to ensure that their message buffers are
present when an IPC transfer occurs. Pagers will typically consider pages
containing these buffers to be pinned, and never unmap them while a thread is
running, unless that thread is expected to be able to handle the resulting
truncation of all of its messages.

\subsubsection{Capability Registers}

The message words can be used to transfer arbitrary untyped data. However, since seL4 capabilities are opaque to user threads, a user thread cannot use message words to transfer capabilities to another thread. The kernel provides a small number of \emph{capability registers} in the IPC buffer; if there are valid capability pointers in those registers, and the sender indicates that the message includes capabilities, then the kernel will send the capabilities along with the message. There are two slightly different motivations for sending a capability along with a message; are two corresponding mechanisms for receiving them at the other end.

An obvious reason to send a capability is to allow the receiver to invoke it in future. For example, a client may send a notification capability to a server, to be used for notifications when asynchronous operations complete; or a server may return a capability representing a newly created object. For such capabilities, the receiver must specify a CSpace slot that the capability can be copied into; this is done by setting fields in the IPC buffer that correspond to the destination arguments of a CNode Copy operation (see \autoref{sec:overview.cspace.cnodes}).

Sometimes, however, a capability is sent solely to prove to the receiver that the sender holds it. This is the case when an operation is being performed that affects two or more objects that are implemented by the same server. In this case, there is no reason to transfer real copies of the capability arguments to the receiver. Therefore, if any of the capability arguments refer to the same endpoint that is being used to send the message, the kernel copies the capability's badge into the receiver's corresponding capability register instead of copying the capability itself.

Note that client threads need not be aware of whether two objects are implemented by the same server, or of which transfer mechanism is used for a given call --- the kernel decides which mechanism to use for each capability argument, and informs the server of these choices. This means that objects conforming to an interface can be transparently distributed between servers.

Both types of capability transfer may be restricted by modifying the rights on the endpoint capabilities. The sender must have the grant right to allow either type of capability transfer; also, if the receiver does not have the send right on the endpoint, then any capability copied to it via IPC will be made read-only.

\subsubsection{Non-Blocking Invocations}

In some instances, a thread may need to send a message through an endpoint
provided by another, untrusted, thread. In this situation, the sender must not
assume that the endpoint capability is valid, or that there will ever be a
partner available to receive the message. To avoid making those assumptions,
the sender may use the non-blocking send operation, \emph{NBSend}.

A non-blocking invocation
will not fail or fault if the invoked capability is absent, and will not
suspend the sender if no partner is waiting on the endpoint. It is the
untrusted recipient's responsibility to provide a valid endpoint capability
and to perform a Recv operation on it before the message is sent; otherwise
the message will be silently dropped. The kernel does not inform the sender
that the non-blocking invocation has been dropped, as doing so would open a
covert channel.

\subsubsection{Calls and Replies}

IPC is often used to implement remote procedure calls (or \emph{RPC}). In order to simplify the process of sending and replying to RPC, the kernel provides special versions of the Send and Recv operations designed specifically for use on the client and server sides of an RPC interface.

To send an RPC, the client marshals arguments in the same way as for an ordinary Send operation, and then performs a \emph{Call} operation. The only difference between Send and Call is that when the message is transferred, a Call will block the sending thread, create a \emph{resume capability} that wakes the sender when invoked, and transfer it to a special slot belonging to the receiving thread: the \emph{caller slot}. This transfer is subject to the same restrictions as an explicit capability transfer: it only happens if the sender possesses the grant right for the endpoint, and the receiver must have the send right on the endpoint or it will receive a read-only capability (which in this case is effectively a null capability).

The caller slot does not exist in the receiver's CSpace. Instead, the receiver can either invoke it using a special \emph{Reply} operation, or move its contents into the CSpace using a CNode method called \emph{SaveCaller}, and invoke it later using a normal Send call. In either case, the resume capability only ever exists in one place --- it cannot be copied --- and is deleted as soon as the calling thread is resumed.

When the resume capability is invoked --- either when the receiver uses Reply, or when some thread sends a message to the capability after the receiver has moved it using SaveCaller --- a message transfer is immediately performed between the replying thread and the blocked client thread. The client thread is then resumed, which has a side-effect of deleting the resume capability.

The Reply operation is always non-blocking. If the resume capability is not valid --- because the sender used Send rather than Call, the sender has already been resumed, the resume capability has been saved in the CSpace, or even if there has never been a sender yet --- then the Reply operation does nothing.

Servers generally run a loop which waits for a message, handles the received message, sends a reply and then returns to waiting. To avoid excessive switching between user and kernel modes, the kernel provides a \emph{ReplyRecv} operation, which is simply a Reply followed by Recv.

\subsection{System Calls}

\autoref{fig:clientkernel} shows a request being made of the kernel by a
client thread. The client possesses a capability to a kernel object, and a
designated \emph{reply capability}, which is an endpoint capability that the
client (or a system thread controlling the client) has nominated to be used
for receiving replies to requests. It performs a Call or Send operation to the kernel
object, which transfers a message to the kernel along with a reference to the
invoked kernel object.

\begin{figure}
\centering \includegraphics{figures/clientkernel}
\caption{A request by a client for a kernel service. The dotted lines
represent the conceptual flow of control and data during the
request.}
\label{fig:clientkernel}
\end{figure}

After receiving a message from a client thread, the kernel determines which
operation is being requested, based on the invoked object's type and the
contents of the message; it then performs the operation if possible. If the user sent the request using a Call operation (rather than Send), the kernel replies with a message that either indicates that the operation succeeded, or gives the reason it did not succeed.

\begin{figure}
\centering \includegraphics{figures/clientserver}
\caption{A request by a client for a service provided by a user-level server.}
\label{fig:clientserver}
\end{figure}

\autoref{fig:clientserver} shows the procedure followed when a client
requests a service provided by a user-level server thread. Rather than invoke
a kernel object capability, the client invokes an endpoint which represents
the service; the server thread receives the message, determines the identity
of the client using the sending capability's badge, performs the operation,
and sends the result to the client's reply capability. Generally, a server
will possess a single capability used to receive requests, and also one reply
capability for each of its clients; a client will have a single reply
capability, and one capability for each object or server it is permitted to
access.

\subsection{System Call Monitors}\label{sec:sel4:monitors}

Client-server communication and client-kernel communication are quite similar.
In fact, if the scheduler can guarantee that the client will never be
scheduled while the operation is running, the two are indistinguishable from
the client's point of view. This allows kernel operations to be intercepted by
a \emph{system call monitor} thread, by replacing the client's kernel object
capabilities with capabilities to an endpoint which the monitor can receive
from. The monitor may then selectively deny, log or modify each system call
before forwarding it to the kernel, possibly logging or forging the kernel's
reply as well.

Note that capability references are sometimes passed to the kernel as system
call arguments, in which case the monitor will need to replace the references
with valid ones in the monitor's address space.

\section{Untyped Memory and Kernel Memory Allocation}\label{sec:sel4:alloc}

The kernel memory management issues with existing L4 kernels are addressed in
seL4 by managing \emph{all} dynamic memory allocation at user level. The
kernel never allocates memory after user-space code begins running.

The CSpace of the initial user-space thread includes, among other things, a
set of capabilities to \emph{untyped memory objects} representing all of the
system's unallocated physical memory. Initially most of these capabilities
represent large regions of memory --- they may represent any power of two
size, up to the size of the entire unallocated region. The initial thread is
free to allocate these regions for specific purposes, or to hand out the
capabilities to other user-space threads that it creates.

To allocate objects in an untyped memory region, the user-level thread that
possesses it must send a message to it, specifying the type of object to
create, the size of the created objects (for object types with variable
sizes), and a capability reference to an area of a CSpace that will be used to
store capabilities to the newly allocated objects. This operation is called
\emph{Retype}. Retype's effect on CSpace is to fill a contiguous empty CSpace
region specified by the caller with new capabilities. If the operation is
successful, the kernel returns the number of capabilities created. The
construction of CSpace is discussed further in \autoref{sec:overview.cspace}.

Allocated objects may be \emph{deactivated} by the \emph{Revoke} or
\emph{Delete} operations (discussed in \autoref{sec:overview.cspace.cnodes}),
which delete capabilities. If either of these operations causes all existing
capabilities to a specific object to be deleted, then the kernel will abort any
operations involving that object and remove all references to it as a typed
object. It is then no longer possible to use that object, until the memory is
reused to create a new object.

If the kernel were to allow a region of memory to be simultaneously used by two
different objects, it might be possible for a malicious or buggy thread to gain
direct access to the contents of a kernel data structure. Therefore, the
Retype operation will fail with a "RevokeFirst" error if there are already
existing objects in the region. The caller should call Revoke on the Untyped
object's capability first, if there is any possibility that it already contains
other objects.

Each Retype operation creates kernel objects of one of the possible types. The
types present in all implementations are:

\begin{description}
\item [Untyped memory] objects, which can be created with a Retype operation
in order to split up a large memory region;
\item [Virtual memory] objects, which can be mapped in the virtual address
space (using some implementation-specific process as described in
\autoref{sec:overview.vm});
\item [Thread Control Block] objects, which each provide the resources needed
for a single user-level thread (see \autoref{sec:overview.threads});
\item [CNode] objects, which are nodes in the guarded page tables used to
translate CSpace addresses (see \autoref{sec:overview.cspace});
\item [Endpoint] objects, which contain queues for synchronous IPC (as
described in \autoref{sec:overview.caps.ipc}); and
\item [Notification] objects, which contain message buffers for
signals (see \autoref{sec:overview.signals}).
\end{description}

There may also be implementation-specific object types. The untyped
memory, virtual memory and CNode objects may be of variable size (limited to
powers of two), though the minimum and maximum sizes are specific to each
implementation and type. Other objects are of fixed, implementation-specific
size.

\section{Threads}\label{sec:overview.threads}

Unlike previous versions of L4, threads in seL4 have no global identifier
visible outside the kernel. A thread is identified only by the CSpace address
of a capability that refers to its \emph{thread control block}, or \emph{TCB}.
A thread may or may not possess a capability to its own TCB; in a typical
system, most threads would not.

There are several operations provided by TCB objects, which can be used to
manipulate the state of the corresponding thread:

\begin{description}
\item [CopyRegisters] is used for modifying the state of a thread; it is similar to the old L4 \emph{ExchangeRegisters} operation. It is given an additional capability argument, which must refer to a TCB that will be used as the source of the transfer; the invoked thread is the destination. The caller may select which of several subsets of the register context will be transferred between the threads. The operation may also suspend the source thread, and resume the destination thread.

There are always at least two subsets of the context that may be copied. The first contains the parts of the register state that are used or preserved by system calls, including the instruction and stack pointers, and the argument and message registers. The second contains the remaining integer registers. Other subsets are architecture-defined, and typically include coprocessor registers such as the floating point registers.

Note that many integer registers are modified or trashed by system calls, so it is not generally useful to use CopyRegisters to copy integer registers to or from the current thread. The kernel therefore provides two specialised operations for those purposes; they are described below.

\item [ReadRegisters] is a variant of CopyRegisters for which the destination is the calling thread. It uses the message registers to transfer the two subsets of the integer registers; the message format has the more commonly transferred instruction pointer, stack pointer and argument registers at the start, and will be truncated at the caller's request if the other registers are not required. It also does not have an option to restart the destination thread, since it is obviously already running.

\item [WriteRegisters] is a variant of CopyRegisters for which the source is the calling thread. It uses the message registers to transfer the integer registers, in the same order used by \emph{ReadRegisters}. It may be truncated if the later registers are not required; an explicit length argument is given to allow error detection when the message is inadvertently truncated by a missing IPC buffer.

\item [SetPriority] configures the thread's scheduling parameters. In the current version of seL4, this is simply a priority for the round-robin scheduler.

\item [SetIPCBuffer] configures the thread's local storage, particularly the IPC buffer used for sending parts of the message payload that don't fit in hardware registers.

\item [SetSpace] configures the thread's virtual memory and capability address spaces. It sets the roots of the trees (or other architecture-specific page table structures) that represent the two address spaces, and also nominates the endpoint that the kernel uses to notify the thread's pager of faults and exceptions.

\item [Configure] is a batched version of the three configuration system calls: SetPriority, SetIPCBuffer, and SetSpace. This is the approximate equivalent of the \emph{ThreadControl} system call in earlier versions of L4.

\item[Resume] will resume execution of a thread that is inactive or waiting
for a kernel operation to complete. If the invoked thread is waiting for a
kernel operation, Resume will modify the thread's state so that it will attempt
to perform the faulting or aborted operation again. Resuming a thread that is
already ready has no effect. Resuming a thread that is in the waiting phase of
a Call operation may cause the sending phase to be performed again, even if
it has previously succeeded.

A CopyRegisters or WriteRegisters operation may optionally include a Resume operation on the destination thread.

\item[Suspend] will make a thread inactive. The thread will not be scheduled
again until a Resume operation is performed on it.

An CopyRegisters or ReadRegisters operation may optionally include a Suspend operation on the source thread.
\end{description}

\section{Capability Address Spaces}\label{sec:overview.cspace}

The capability space is represented by a \emph{guarded capability table},
which is similar to a guarded page table~\cite{Liedtke-94a}. The nodes of the
table are individual kernel objects allocated using the Retype system call,
and are called \emph{capability nodes}, abbreviated \emph{CNodes}.

Each CNode has a power-of-two number of \emph{slots}; the number is determined
at the time the CNode is created. Each slot contains a \emph{capability table
entry}, which can store a single capability; the contents of the entry are
opaque to user-space code. The format of entries is the same at all levels of
the table. There is also a slot in each TCB, which is used to store a
capability to the root CNode.

The capability table is constructed by copying a capability to the root CNode
into the TCB's root slot, capabilities to lower-level CNodes into the root
CNode's slots, and so on. The number of address bits resolved by each CNode
depends on the node's size and on the number of \emph{guard bits} of the CNode.
A capability will appear in the user thread's CSpace when it can be reached by
translating address bits starting at the page table root capability.

Note that all CNodes are required to resolve at least one address bit, to
prevent infinite loops in the capability lookup code.

\subsection{The Mapping Database}\label{sec:overview.cspace.mdb}

The kernel keeps a record of the \emph{derivation tree} for capabilities. This
tree is stored in the \emph{mapping database}, and is used when revoking a
capability to locate the capabilities which must be deleted. This is similar to
previous L4 kernels, except that the storage for the mapping database is not
dynamically allocated in seL4 --- every capability table slot includes some
space reserved for a mapping database node.

For implementation reasons, the kernel imposes limitations on the structure of
the derivation tree. In particular, the tree may not have unlimited depth. The
exact nature of these limitations is implementation-defined.

The mapping database is optional. With some minor modifications to the API, it
can be removed from the kernel entirely; this halves the size of a CNode slot,
and therefore considerably reduces kernel resource usage. However, it also
prevents the kernel from implementing the Revoke operation, and makes multiple
uses of the Retype operation on a single memory region unsafe. Therefore its
removal is only appropriate for systems that never need to revoke or reuse
system resources, or that completely trust any user-level thread that has
access to untyped memory.

This document describes the API for kernels that have a mapping database.

\subsection{CNode Operations}\label{sec:overview.cspace.cnodes}

User-level tasks possessing a capability to a CNode may perform any of five
operations on the capabilities stored in the tree of which that CNode is the
root. Each operation modifies at least one capability slot in the tree; the
caller must specify an address within the tree and the number of significant
bits in that address.

\begin{description}
\item[Mint] creates a new capability in a specified CNode slot, given a
reference to a slot containing an existing capability. The newly created
capability may differ from the original --- it may have fewer rights, and its
type-specific data (such as the IPC badge on an endpoint capability) may be
changed ---  but it will always refer to the same kernel object.

If the source capability is an endpoint, it must not have previously had
new capability data set; otherwise, use of this operation will not be permitted.

\item[Copy] is similar to Mint, but does not change the capability's
type-specific data.

\item[Move] moves a capability between two specified capability slots. It does
not change the rights on the capability, but it may change type-specific data
on capabilities other than endpoints.

\item[Rotate] moves two capabilities between three specified capability slots. It is essentially two Move operations; one from the second specified slot to the first, and one from the third to the second. The first and third specified slots may be the same, in which case the capability in it is swapped with the capability in the second slot. The operation is atomic; either both capabilities are moved successfully, or neither is moved.

\item[Delete] will remove a capability from the specified slot. If the capability is the last one that refers to a given kernel object, the object will be \emph{destroyed}, which removes any references to it in other parts of the kernel and makes the memory storing it safe for reuse.

\item[Revoke] is equivalent to calling Delete on each derivation tree
child of the specified capability. It has no effect on the capability itself.

\item[Recycle] is equivalent to Revoke, except that it also returns the object to its initial state. It may only be performed on the original capability to a typed object.

\item[SaveCaller] moves a kernel-generated reply capability from the special TCB slot it was created in, into the designated CSpace slot.
\end{description}

Delete, Revoke and Recycle are potentially very long-running operations, especially if:
\begin{itemize}
\item a capability is revoked while it has a large number of children in the
derivation tree, or
\item a CNode is recycled, or the last capability to a CNode is destroyed (either by a direct Delete call, or a Revoke call on the Untyped capability controlling the CNode's storage), while that CNode still contains valid capabilities.
\end{itemize}

Systems with real-time constraints should avoid using Revoke, Recycle or Delete unless
they can ensure that only a small number of capabilities will be deleted.
Implementations of the seL4 API may include preemption points in this operation; the points at which preemption is safe are indicated in the Haskell code by "preemptionPoint" calls. Note that these preemption points allow certain operations to fail non-deterministically --- specifically, if any capabilities that were used to request the operation are deleted by it, then the operation will return a missing capability error if it is preempted. Deliberate execution of such an operation is not recommended, though it is unlikely to ever be necessary.

Copy, Mint, Move and Retype operations require their destination slot
or slots to be empty; that is, they must not contain valid capabilities. If any
of these operations finds a valid capability in a destination slot, it will
fail with a "DeleteFirst" error. As the error name suggests, the caller should
call Delete on destination slots first, if there is any possibility that they
might contain valid capabilities. This restriction also applies to the destination slot for Rotate, unless it is the same as the second source slot.

\subsection{Capability Rights}\label{sec:overview.cspace.rights}

The set of operations which may be performed on a given capability are
determined by the type of the object that the capability refers to, and a set
of \emph{rights} specific to the capability. The rights are a small set of bits corresponding to classes of operation:

\begin{description}
\item[Write] allows data to be sent or written to the capability's object. For
example, an endpoint capability must have this right if it is to be used to
send messages. If an endpoint capability that does not have this right is used to receive other capabilities, the received capabilities will be \emph{diminished} by removing this right.

\item[Read] allows data to be received or read from the object.

\item[Grant] allows capabilities to be sent to the object. Endpoint capabilities must have this right to be used for transferring capabilities via IPC.
\end{description}

Newly created capabilities have all rights. They may be reduced when performing
Copy, Mint or Move operations, either via a CSpace invocation or a capability
transfer IPC.

\section{Faults and Exceptions}\label{sec:overview.faults}

Whenever a user-level task raises a fault, the kernel will suspend
it and send a \emph{fault call} to the thread's \emph{fault handler
capability}. The latter is a capability reference stored in the thread's TCB,
and can be set or changed using the ThreadControl system call. The faulting thread will remain suspended until the recipient of the fault call replies to the call; if the fault handler endpoint is not valid, it will remain suspended indefinitely.

The fault message contains information about the nature and cause of the
fault, sufficient for the recipient of the fault call to recover from the
fault if possible. The recipient may take any appropriate action to handle the fault. If it replies to the fault call, the kernel will adjust the state and context of the thread depending on the content of the reply.

Events that generate fault calls include missing capabilities, missing virtual memory mappings, non-seL4 system calls, and architecture-defined exceptions such as illegal instructions. The content of the fault calls, and the semantics of the replies, depends on the type of fault. Definitions for specific types of fault can be found in \autoref{sec:api.faults}

\section{Virtual Address Spaces}\label{sec:overview.vm}

Many modern hardware architectures dictate specific page table formats to be
used by the kernel to implement virtual memory. These usually require the
dynamic allocation of memory to be used for page tables; since the seL4 kernel
never dynamically allocates its own memory, the page tables must therefore be
exposed to user level somehow.

The kernel's interface for virtual memory management is
implementation-defined, but will generally fit into one of two categories:
architectures with hardware-loaded TLBs and hardware-defined page table
structures, and those with software-loaded TLBs.

\subsection{Software Loaded TLBs}

On architectures with software-loaded TLBs, the kernel simply uses a structure
similar to that of CSpace, constructed from CNodes, to implement the virtual
address space. There is a separate root node in each TCB --- though it may
contain a capability to a CNode that is also present in the CSpace, if the
system requires CSpace and VSpace addresses to be consistent.

\subsection{Hardware Loaded TLBs}

On architectures with hardware-defined page table structures, such as ARM and
x86, a separate address space is used in parallel with CSpace. This address
space is known as \emph{VSpace}, an abbreviation of \emph{virtual address
space}.

The details of the VSpace interface are architecture-dependent. Typically they
include one or more new object types, from which a page table can be
constructed in a manner similar to the construction of CSpace. Messages sent
to the new objects are used to request operations similar to those provided by
CNode objects. As the source parameters of such operations are always
capabilities (in the CSpace), it is not possible to copy mappings between
different locations in VSpace without possessing an appropriate capability.

\section{Signals}\label{sec:overview.signals}

In some situations, it is appropriate to provide an asynchronous notification
of an event. The kernel provides a \emph{signal} operation for this
purpose. It is implemented using \emph{notification} objects, which
contain a fixed size buffer of two machine words -- identifier and message word.
Each notification capability is marked with a badge, which is set when
the capability is created. A thread may perform \emph{Send} or \emph{Recv}
operations on a notification, by invoking the corresponding
capability.

When a Send operation is performed, the badge of the invoked capability is bitwise ORed with the corresponding values in the
endpoint to produce its new contents. This operation is guaranteed not to block
a sufficiently authorised sender, even if there is no thread waiting to receive
the message. The Recv operation, on the other hand, can be blocking --- unless the operation is specifically non-blocking, if the
buffer is empty, the thread is blocked until a message arrives. However, in
case of a non-empty buffer, the Recv operation will read the contents of the
buffer, reset it to zero, and returns the value immediately.

Similar to synchronous endpoints, multiple threads may perform Signal or Recv
operations on the same notification. When there are multiple messages
sent to an endpoint before any thread attempts to receive from it, the
identifier and the message delivered to the receiver are the bitwise OR of
capability badges used by the senders and the messages sent by them,
respectively. It is the responsibility of the user level tasks to define a
suitable protocol for identifying the individual event. When multiple receivers
are waiting on the same endpoint when a message is sent to it, the kernel
chooses one of them to deliver the message. The algorithm for choosing a thread
is not specified, but typically will be first in, first out order.

\section{Interrupts}\label{sec:overview.interrupts}

Interrupt handling in seL4 is performed using two special types of kernel
capability. These capabilities cannot be created using the Retype operation,
because they do not represent dynamically allocated memory resources.

The system has a single global \emph{IRQ control} capability, which is provided
to the initial user-level task at boot time. Copies of this capability may be
made in the usual fashion, but every copy permits access to the same global
state.

The IRQ control capability is used to create \emph{IRQ handler} capabilities.
Each IRQ handler capability refers to a single hardware IRQ line, specified when
the capability is created; creation only succeeds if the specified IRQ is valid
and has no existing handler (including any internal handler in the kernel, such
as for the timer interrupt).

The IRQ handler capability is used to set the mechanism used by the kernel for
dispatching interrupts on the corresponding IRQ line, and to enable and disable
the IRQ. The capability's IRQ is enabled when a delivery mechanism is set, or
when the \emph{Ack} operation is performed. It is masked when the last handler
capability for the IRQ is deleted, when an interrupt is delivered, or when the
delivery mechanism becomes invalid (due to a \emph{Clear} operation on the IRQ
handler capability, or revocation of a capability required for delivery).

In the present Haskell implementation, only one delivery mechanism is supported:
delivery via signal to a specified notification. This may be complemented
or replaced by a callback-based delivery model in future versions or
real-hardware implementations.




\chapter{Haskell Model Overview}\label{sec:haskell}
%
% Copyright 2014, General Dynamics C4 Systems
%
% This software may be distributed and modified according to the terms of
% the GNU General Public License version 2. Note that NO WARRANTY is provided.
% See "LICENSE_GPLv2.txt" for details.
%
% @TAG(GD_GPL)
%

This chapter describes the construction of the Haskell model presented in the following chapters. It is not necessary to read this chapter to understand the seL4 API, but it will be helpful for those who wish to read the Haskell code.

\section{Introduction to Haskell}\label{sec:haskell.intro}

Haskell is a general purpose functional programming language~\cite{peytonjones03haskell}. It is in widespread use in the research and education communities; several universities use it in introductory programming courses.

The features of Haskell that are most relevant to this project are:
\begin{itemize}
\item The type system performs strict checks at compile time; it must be possible to determine the type of every expression. Invalid values such as null pointers are impossible, as are unsafe or implicit type conversions. This simplifies debugging, as most incorrect code will not compile.
\item There is a formal definition of the language's semantics~\cite{Fax:static}. There is only one case in which it is ambiguous; that case is unlikely and easily avoided. The pure functional semantics and strict typing make Haskell quite similar to HOL, which is being used in the ongoing L4 verification project~\cite{Tuch-KH-05}.
\item By using constructs called \emph{monads}, it is possible to write functions that process state changes in a manner that superficially resembles an imperative language. These are easier to read than non-monadic functional programs when performing complex and inherently imperative state transformations, especially for developers who have no experience with functional languages. This is discussed further in section \ref{sec:haskell.state-monads}.
\end{itemize}

A more detailed introduction to Haskell is outside the scope of this document. Please refer to the online Haskell tutorials~\cite{haskell-tutorial,monad-tutorial} for more information.

\section{Modules}\label{sec:haskell.modules}

The Haskell model is split into a hierarchy of modules. At the top level is "module SEL4", which contains no real code, but exports the entire external interface of the kernel model. Below this are five modules that serve to separate the high-level and low-level parts of the model; each of these corresponds to one of the following five chapters in this report.

The five modules have the following purposes:
\begin{description}
\item[API] defines the interface between the kernel and the user-level threads running under it (other than the parts relating to specific kernel objects). See \autoref{sec:api}.
\item[Kernel] defines internal procedures of the kernel that do not relate to a specific kernel object --- most importantly, the scheduler and the capability space lookup functions. See \autoref{sec:kernel}.
\item[Object] defines the first-class kernel objects exposed by the seL4 API, and the operations that may be performed on them. See \autoref{sec:object}.
\item[Model] contains the implementation of the low-level parts of the Haskell kernel --- the physical memory model and the kernel state monads. See \autoref{sec:model}.
\item[Machine] contains the kernel's interface to the underlying hardware --- specifically the register set, the kernel's uses for each register, the word type, and the virtual page size. See \autoref{sec:machine}.
\end{description}

Each of these modules is further divided into smaller modules implementing separate functional areas. For example, "module SEL4.Kernel" contains a module implementing capability space lookups, "module SEL4.Kernel.CSpace".

\section{System State and Monads}\label{sec:haskell.state-monads}

\subsection{The State Monad}

Haskell is a \emph{pure} functional language, meaning that expressions in the language must neither depend on the global state of the system, nor change it as a side effect\footnote{There are a few exceptions to this rule, including functions in the {\hsfamily\scshape IO} monad and the foreign function interface. These are special cases, and can only be used under specific conditions.}. This makes it fundamentally very different to imperative languages such as C, which are typically used for kernel implementations.

One problem with pure functional programming is that systems that operate in a complex state space --- for example, models of operating system kernels --- must pass their entire state around in function parameters and return values. This clutters the code and results in programs that are difficult to read, especially for people unfamiliar with functional programming. As a trivial example, this is a function for which the state data is an integer; it adds a given value to the state, and returns the new value converted to a string:

\functions (updateAndShow)

> updateAndShow :: Integer -> Integer -> (Integer, String)
> updateAndShow step state = (new_state, show new_state)
>         where new_state = state + step

Note that the function must both accept and return an extra "Integer" value representing its state. Also, the entire result must be constructed in one expression. If the computation conceptually involves a sequence of imperative steps, representing it this way can be quite difficult.

However, Haskell includes support for \emph{monadic} programming~\cite{wadler92essence}, and provides a \emph{monad} called "State". This monad provides a means to hide the explicit state parameters, and express sequences of state transitions in a form that superficially resembles an imperative program. To repeat the previous example using "State":

\functions (get, put)

> updateAndShow :: Integer -> State Integer String
> updateAndShow step = do
>         old_value <- get
>         let new_value = old_value + step
>         put new_value
>         return (show new_value)

The "get" and "put" functions used in this example are for fetching and setting the current state. Note that due to Haskell's strict typing requirements, transitions between functions that are not in the "State" monad and those that are must be explicit --- using the "runState" function to evaluate expressions in "State" from outside the monad, and the "let" statement to evaluate non-monadic expressions from inside the monad.

For complex state transformations, writing programs in monadic style provides a significant improvement in readability over traditional functional programming. Monadic style has been used extensively in this specification. Nearly all of the functions in the model are in "KernelMonad", which is an application of the "State" monad to the "KernelState" type\footnote{In fact, {\hsfamily\scshape KernelMonad} is an application of the {\hsfamily\scshape StateT KernelState} monad transformer to another monad that represents the state of the hardware; the exact type of the latter depends on which simulator is being attached to the model. The result can be used as if it were the {\hsfamily\scshape State KernelState} monad, except that it also allows access to the underlying hardware state monad.}. Both are defined in \autoref{sec:model.statedata}. The "KernelState" type is described briefly in \autoref{sec:haskell.state-monads.system-state}.

\subsection{Errors and Faults}

One limitation of the "State" monad is that there is no straightforward way to interrupt processing of a sequence of operations when an error occurs. The "ErrorT" monad transformer provides a mechanism for doing so.

Using "ErrorT" allows sequences of computations in a monad to fail and return an error value. Further expressions in the sequence will not be evaluated; the error will be passed up the call stack in a manner that resembles a C++ or Java exception, until it reaches a call to "catchError". For example, to extend the "updateAndShow" function defined above so it returns an error message if the "step" is not positive:

\functions (throwError, unless, increaseAndShow)

> increaseAndShow :: Integer -> ErrorT String (State Integer) String
> increaseAndShow step = do
>         unless (step > 0) $ throwError "step must be positive"
>         lift $ updateAndShow step

Again, Haskell's strict typing requires all transitions in and out of the "ErrorT" monad transformer to be explicit. The "runErrorT" function enters the monad; "let" evaluates a non-monadic expression; and the function "lift" adds the "ErrorT" transformer to the type of a function that is already in "State".

The Haskell kernel model makes use of the "ErrorT" transformer to abort operations that are preempted or cannot complete successfully. The first type parameter of "ErrorT" is a type that represents the error; there are several such types defined in seL4, in \autoref{sec:model.failures} and \autoref{sec:model.preemption}.

\subsection{System State}
\label{sec:haskell.state-monads.system-state}

The state of the simulated kernel is stored in values of type "KernelState". Most functions in the model are in the monad "KernelMonad", and are therefore state transformation functions that operate on a value of type "KernelState".

The state data structure contains a value of type "PSpace", which is used to store data which is dynamically allocated by calls from user level. See \autoref{sec:haskell.state-monads.pspace} for details.

The other values present in the "KernelState" structure are global kernel data, such as a pointer to the current thread.

The "KernelState" value is added to "KernelMonad" by a monad transformer, "StateT". This adds the state to an existing monad, the \emph{machine monad}, the type of which depends on the machine being simulated. See \autoref{sec:haskell.simulator.monad}.

\subsection{Physical Memory Model}\label{sec:haskell.state-monads.pspace}

The simulated kernel's physical memory is modelled by the Haskell type "PSpace", defined in \autoref{sec:model.pspace}. The contents of each address in this space are typed; that is, the model keeps a record of whether they are allocated, and whether they contain either integer data, or a specific type of kernel object. Accesses to objects of the wrong type, or with an incorrectly aligned address, are detected by the model and disallowed.

For simulator performance reasons, and to make the operation of the virtual memory related parts of the model more realistic, the "PSpace" contains only typed kernel objects; it does not contain data accessible to user level. Instead, regions occupied by such data are represented by a "UserData" object, which has no contents. The actual data is stored separately, and is accessed via the machine monad.

\section{The Simulator}\label{sec:haskell.simulator}

The kernel model functions as one half of a complete system model. The other half is the CPU simulator, responsible for modelling the execution of user-level programs and the operation of hardware devices.

\subsection{Events}

When the state of the simulated system changes in a way that the kernel must respond to, the simulator sends the kernel an \emph{event}. These events correspond to situations that would cause kernel code to execute on a real system: specifically hardware exceptions and interrupts, and user-level software traps.

The set of events supported by the Haskell model is defined by the type "Event", found in \autoref{sec:api.types}.

\subsection{Hardware Parameters}

Several aspects of the kernel's API and behaviour vary depending on the architecture of the host machine. The model contains abstract interfaces to the components that change; the interfaces are implemented in separate modules, selected using the preprocessor.

\subsection{Simulator Interface}\label{sec:haskell.simulator.monad}

The state of the simulated machine is accessed via a monad, the type of which depends on which CPU simulator is in use --- again, selected using the preprocessor. For example, for a simulator implemented entirely in Haskell, this monad is likely to be "State", with a data structure containing the machine's state data. If the simulator is implemented externally, and communicates with the kernel via the Haskell foreign function interface, then the machine monad will be "IO" (or some transformation of "IO"), allowing functions in it to call out to the external simulator.

\section{Literate Haskell}

The Haskell code presented in this document is \emph{literate Haskell}; it can be compiled directly by the Haskell compiler, but is also a valid \LaTeX\ document. The code is typeset using the \lambdaTeX\ package, in a {\hsfamily different typeface}, as are Haskell symbols that are mentioned in the text. Data and type constructors, which always begin with a capital letter, are set in {\hsfamily\scshape SmallCaps}, and variable names are set in {\hsfamily\itshape italics}.

Some of the Haskell code is omitted for clarity. In particular, we omit most "import" statements (other than those which are qualified with local module names), compiler directives, and preprocessor directives. We also omit some modules that implement the interface between the kernel model and the external user-level simulators.




\part{Annotated Source Code}

\chapter{Kernel API}\label{sec:api}

This chapter documents the parts of the kernel that interact directly with user-level code.

\section{Kernel Object Types}\label{sec:api.types}
\input{../src/SEL4/API/Types.lhs}

\section{Machine-Independent Object Types}\label{sec:api.types.universal}
\input{../src/SEL4/API/Types/Universal.lhs}

\section{Kernel Object Invocations}\label{sec:api.invocation}
\input{../src/SEL4/API/Invocation.lhs}

\section{System Calls}\label{sec:api.syscall}
\input{../src/SEL4/API/Syscall.lhs}

\section{System Call Failures}\label{sec:api.failures}
\input{../src/SEL4/API/Failures.lhs}

\section{Faults and Exceptions}\label{sec:api.faults}
\input{../src/SEL4/API/Faults.lhs}

\chapter{Kernel Operations}\label{sec:kernel}

This chapter documents operations within the kernel that are not directly related to a single kernel object.

\section{Capability Space Lookups}\label{sec:kernel.cspace}
\input{../src/SEL4/Kernel/CSpace.lhs}

\section{Threads and Scheduling}\label{sec:kernel.thread}
\input{../src/SEL4/Kernel/Thread.lhs}

\section{Bootstrapping the Kernel}\label{sec:kernel.init}
\input{../src/SEL4/Kernel/Init.lhs}

\section{Handling Faults}\label{sec:kernel.faulthandler}
\input{../src/SEL4/Kernel/FaultHandler.lhs}

\section{Virtual Memory}\label{sec:kernel.vspace}
\input{../src/SEL4/Kernel/VSpace.lhs}

\chapter{Kernel Objects}\label{sec:object}

This chapter defines the first-class kernel objects, including their representation in physical memory, and the operations that may be performed on them by user-level threads and by other kernel-level code.

\section{Data Structures}\label{sec:object.structures}
\input{../src/SEL4/Object/Structures.lhs}

\section{Thread Control Blocks}\label{sec:object.tcb}
\input{../src/SEL4/Object/TCB.lhs}

\section{Capability Nodes}\label{sec:object.cnode}
\input{../src/SEL4/Object/CNode.lhs}

\section{Synchronous Endpoints}\label{sec:object.endpoint}
\input{../src/SEL4/Object/Endpoint.lhs}

\section{Notifications}\label{sec:object.notification}
\input{../src/SEL4/Object/Notification.lhs}

\section{Untyped Objects}\label{sec:object.untyped}
\input{../src/SEL4/Object/Untyped.lhs}

\section{Interrupt Controller and Handler Objects}\label{sec:object.interrupt}
\input{../src/SEL4/Object/Interrupt.lhs}

\section{Type-Specific Operations}\label{sec:object.objecttype}
\input{../src/SEL4/Object/ObjectType.lhs}

\section{Storing Objects}\label{sec:object.instances}
\input{../src/SEL4/Object/Instances.lhs}

\chapter{Haskell Model Details}\label{sec:model}

This chapter describes implementation details of the Haskell kernel. It should not be necessary to read this section to understand the API; however, it may help in understanding the way the model works.

\section{System State}\label{sec:model.statedata}
\input{../src/SEL4/Model/StateData.lhs}

\section{Physical Address Space Model}\label{sec:model.pspace}
\input{../src/SEL4/Model/PSpace.lhs}

\section{System Calls}\label{sec:model.syscall}
\input{../src/SEL4/Model/Syscall.lhs}

\section{Failures}\label{sec:model.failures}
\input{../src/SEL4/Model/Failures.lhs}

\section{Preemption}\label{sec:model.preemption}
\input{../src/SEL4/Model/Preemption.lhs}

\chapter{Modelling the Hardware}\label{sec:machine}\label{ch:code:machine}

This chapter defines the interface that is used by the rest of the kernel to represent the underlying hardware.

\section{Words and Registers}\label{sec:machine.registerset}
\input{../src/SEL4/Machine/RegisterSet.lhs}

\section{Hardware Functions}\label{sec:machine.hardware}
\input{../src/SEL4/Machine/Hardware.lhs}

\part{Architecture-Specific Features}

\chapter{ARM}

This chapter defines the implementation-specific parts of the API for the ARM architecture. The ARM support is designed for version 6 or later of the ARM architecture. In particular, it uses the ARMv6 page table format, and does not make use of the Fast Context Switch Extension which is performance-critical on ARMv5 and earlier.

\section{Register Set}
\input{../src/SEL4/Machine/RegisterSet/ARM.lhs}

\section{Hardware Interface}
\input{../src/SEL4/Machine/Hardware/ARM.lhs}

\section{Kernel Object Types}
\input{../src/SEL4/API/Types/ARM.lhs}

\section{Kernel Object Invocations}
\input{../src/SEL4/API/Invocation/ARM.lhs}

\section{Kernel Object Data}
\input{../src/SEL4/Object/Structures/ARM.lhs}

\section{Type-Specific Operations}
\input{../src/SEL4/Object/ObjectType/ARM.lhs}

\section{Storing Objects}
\input{../src/SEL4/Object/Instances/ARM.lhs}

\section{Global Kernel Data}
\input{../src/SEL4/Model/StateData/ARM.lhs}

\section{Interrupt Handling}
\input{../src/SEL4/Object/Interrupt/ARM.lhs}

\section{Virtual Address Space}
\input{../src/SEL4/Kernel/VSpace/ARM.lhs}

\section{Switching Threads}
\input{../src/SEL4/Kernel/Thread/ARM.lhs}

\section{Copying Coprocessor Registers}
\input{../src/SEL4/Object/TCB/ARM.lhs}

\begin{impdetails}

\chapter{Simple Haskell-based Simulator}

This chapter defines a language and interpreter that may be used for writing simple test programs for the seL4 API. The language is intended to look similar to the assembly language of a simple RISC processor. This chapter also defines implementation-specific API features for the simulated machine.

\section{Register Set}
\input{../src/SEL4/Machine/RegisterSet/HaskellCPU.lhs}

\section{Hardware Interface}
\input{../src/SEL4/Machine/Hardware/HaskellCPU.lhs}

\section{API}
\input{../src/SEL4/API/Types/HaskellCPU.lhs}

\section{Kernel Objects}
\input{../src/SEL4/Object/ObjectType/HaskellCPU.lhs}

\section{Virtual Address Space}
\input{../src/SEL4/Kernel/VSpace/HaskellCPU.lhs}

\end{impdetails}

\bibliographystyle{plain}
\bibliography{pubs}

\end{document}


