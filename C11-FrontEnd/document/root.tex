\documentclass[fontsize=11pt,paper=a4,open=right,twoside,abstract=true]{scrreprt}
\usepackage[T1]{fontenc}
\usepackage[utf8]{inputenc}
\usepackage{lmodern}
\usepackage[numbers, sort&compress, sectionbib]{natbib}
\usepackage{isabelle,isabellesym}
\usepackage{ifthen}
\IfFileExists{railsetup.sty}{\usepackage{railsetup}}{}

%%%%%%%%%%%%%%%%%%%%%%%%%%%%%%%%%%%%%%%%%%%%%%%%%%%%%%%%%%%%%%%%%%%%%%%%%%%%%%%
% command

\newenvironment{matharray}[1]{\[\begin{array}{#1}}{\end{array}\]} % from 'iman.sty'
\newcommand{\indexdef}[3]%
{\ifthenelse{\equal{}{#1}}{\index{#3 (#2)|bold}}{\index{#3 (#1\ #2)|bold}}} % from 'isar.sty'

%%

%%%%%%%%%%%%%%%%%%%%%%%%%%%%%%%%%%%%%%%%%%%%%%%%%%%%%%%%%%%%%%%%%%%%%%%%%%%%%%%
% fix for package declaration to be at the end
\usepackage[pdfpagelabels, pageanchor=false, plainpages=false]{hyperref}

%%%%%%%%%%%%%%%%%%%%%%%%%%%%%%%%%%%%%%%%%%%%%%%%%%%%%%%%%%%%%%%%%%%%%%%%%%%%%%%
% document

\urlstyle{rm}
\isabellestyle{it}

\begin{document}

\title{Isabelle/C}
\author{%
  \href{https://www.lri.fr/~ftuong/}{Fr\'ed\'eric Tuong}
  \and
  \href{https://www.lri.fr/~wolff/}{Burkhart Wolff}}
\publishers{%
  \mbox{LRI, Univ. Paris-Sud, CNRS, CentraleSup\'elec, Universit\'e Paris-Saclay} \\
  b\^at. 650 Ada Lovelace, 91405 Orsay, France \texorpdfstring{\\}{}
    \href{mailto:"Frederic Tuong"
    <frederic.tuong@lri.fr>}{frederic.tuong@lri.fr} \hspace{4.5em}
    \href{mailto:"Burkhart Wolff"
    <burkhart.wolff@lri.fr>}{burkhart.wolff@lri.fr}
}

\maketitle

\begin{abstract}
Isabelle/C is a C front-end for Isabelle (currently version C11), pretty much resembling to the
Isabelle/ML language, except that the given C code is parsed and sent to \emph{semantic back-ends}
for more sophisticate evaluation treatments.\cite{brucker.ea:featherweight:2014}
\end{abstract}
\tableofcontents

\parindent 0pt\parskip 0.5ex

%%%%%%%%%%%%%%%%%%%%%%%%%%%%%%%%%%
\input{session}

%%%%%%%%%%%%%%%%%%%%%%%%%%%%%%%%%%

\bibliographystyle{abbrvnat}
\bibliography{root}

%%%%%%%%%%%%%%%%%%%%%%%%%%%%%%%%%%

\end{document}

%%% Local Variables:
%%% mode: latex
%%% TeX-master: t
%%% End:
