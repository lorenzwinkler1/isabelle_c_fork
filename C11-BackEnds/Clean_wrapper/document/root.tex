\documentclass[fontsize=11pt,paper=a4,open=right,twoside,abstract=true]{scrreprt}
\usepackage[T1]{fontenc}
\usepackage[utf8]{inputenc}
\usepackage{lmodern}
\usepackage[numbers, sort&compress, sectionbib]{natbib}
\usepackage{isabelle,isabellesym}
\usepackage[only,bigsqcap]{stmaryrd}
\usepackage{ifthen}
\usepackage{wrapfig}
\usepackage{graphicx}
\usepackage{xcolor}
\usepackage{listings}
\IfFileExists{railsetup.sty}{\usepackage{railsetup}}{}

\newcommand\inputif[1]{\IfFileExists{./#1}{\input{#1}}{}}

%%%%%%%%%%%%%%%%%%%%%%%%%%%%%%%%%%%%%%%%%%%%%%%%%%%%%%%%%%%%%%%%%%%%%%%%%%%%%%%
% command

\newenvironment{matharray}[1]{\[\begin{array}{#1}}{\end{array}\]} % from 'iman.sty'
\newcommand{\indexdef}[3]%
{\ifthenelse{\equal{}{#1}}{\index{#3 (#2)|bold}}{\index{#3 (#1\ #2)|bold}}} % from 'isar.sty'
\newcommand{\isactrlC}{{\isacommand C}}
\newcommand{\isactrlmkUNDERSCOREstring}{\isakeywordcontrol{mk{\isacharunderscore}string}}
\newcommand{\isactrlMLPRIME}{{\isacommand ML{\isacharprime}}}

%%

%%%%%%%%%%%%%%%%%%%%%%%%%%%%%%%%%%%%%%%%%%%%%%%%%%%%%%%%%%%%%%%%%%%%%%%%%%%%%%%
% fix for package declaration to be at the end
\usepackage[pdfpagelabels, pageanchor=false, plainpages=false]{hyperref}

%%%%%%%%%%%%%%%%%%%%%%%%%%%%%%%%%%%%%%%%%%%%%%%%%%%%%%%%%%%%%%%%%%%%%%%%%%%%%%%
% document

\urlstyle{rm}
\isabellestyle{it}

\newcommand{\HOL}{HOL}
\newcommand{\eg}{e.g.}
\newcommand{\ie}{i.e.}

\begin{document}

\title{Isabelle/C/Clean - A Show-Case for a Semantic Back-End for Isabelle/C}
\author{%
  \href{https://www.lri.fr/~ftuong/}{Fr\'ed\'eric Tuong}
  \and
  \href{https://www.lri.fr/~wolff/}{Burkhart Wolff}}
\publishers{%
  \mbox{LRI, Univ. Paris-Sud, CNRS, Universit\'e Paris-Saclay} \\
  b\^at. 650 Ada Lovelace, 91405 Orsay, France \texorpdfstring{\\}{}
}

\maketitle

\begin{abstract}
Isabelle/C~\cite{TuongWolff19} is a C front-end for Isabelle/PIDE providing
generic support for C parsing, editing, and highlighting. Isabelle/C also
provides a generic interface for semantic interpretations of C11 programs and
program fragments. In particular, Isabelle/C also offers the generic framework
to define \emph{annotation commands} and \emph{C antiquotations} that can be
custumized to a specific semantic back-end.

Clean is such a semantic back-end, i.e. an Isabelle theory providing a
semantics and execution model. Clean is an abstract, ``shallow-style
embedding'' for an imperative core language. It strives for a type-safe
notation of program-variables, an incremental construction of the typed
state-space, support of incremental verification, and open-world extensibility
of new type definitions being intertwined with the program definitions.

Clean is based on a ``no-frills'' state-exception monad with the usual
definitions of \isa{bind} and \isa{unit} for the compositional glue of
state-based computations.  Clean offers conditionals and loops supporting
C-like control-flow operators such as \isa{break} and \isa{return}. The
state-space construction is based on the extensible record package. Direct
recursion of C-functions is supported.

The current session instantiates Isabelle/C with Clean and produces the
Isabelle/C/Clean instance. It is conceived as a demonstrator for creating other
instances of Isabelle/C with other semantic backends such as
AutoCorres~\cite{DBLP:conf/pldi/GreenawayLAK14},
IMP2~\cite{DBLP:journals/afp/LammichW19} or Orca~\cite{bockenek:hal-02069705}.
\end{abstract}

\newpage
\tableofcontents

\parindent 0pt\parskip 0.5ex

%%%%%%%%%%%%%%%%%%%%%%%%%%%%%%%%%%

\inputif{1/document/Clean_Wrapper.tex}

\inputif{2/document/Prime.tex}
\inputif{2/document/IsPrime_sqrt_outside.tex}
\inputif{2/document/Quicksort.tex}
\inputif{2/document/Quicksort2.tex}

\inputif{1/document/Init.tex}
\inputif{1/document/Meta_C.tex}
\inputif{1/document/Core.tex}
\inputif{1/document/Isabelle_code_runtime.tex}
\inputif{1/document/Isabelle_typedecl.tex}
\inputif{1/document/Generator_dynamic_sequential.tex}
\inputif{2/document/Clean_backend_old.tex}

%%%%%%%%%%%%%%%%%%%%%%%%%%%%%%%%%%

\bibliographystyle{abbrvnat}
\bibliography{root}

%%%%%%%%%%%%%%%%%%%%%%%%%%%%%%%%%%

\end{document}

%%% Local Variables:
%%% mode: latex
%%% TeX-master: t
%%% End:
