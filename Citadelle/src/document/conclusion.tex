\part{Conclusion}

\chapter{Conclusion}

\section{Lessons Learned and Contributions}
We provided a typed and type-safe shallow embedding of the core of
UML~\cite{omg:uml-infrastructure:2011,omg:uml-superstructure:2011} and
OCL~\cite{omg:ocl:2012}. Shallow embedding means that types of OCL
were mapped by the embedding one-to-one to types in
Isabelle/HOL~\cite{nipkow.ea:isabelle:2002}.  We followed the usual
methodology to build up the theory uniquely by conservative extensions
of all operators in a denotational style and to derive logical and
algebraic (execution) rules from them; thus, we can guarantee the
logical consistency of the library and instances of the class model
construction. The class models were given a closed-world interpretation 
as object-oriented datatype theories, as
long as it follows the described methodology.%
\isatagafp
\footnote{Our two
  examples of \inlineisar+Employee_AnalysisModel+ and
  \inlineisar+Employee_DesignModel+ (see
  \autoref{ex:employee-analysis:uml} and
  \autoref{ex:employee-analysis:ocl} as well as
  \autoref{ex:employee-design:uml} and
  \autoref{ex:employee-design:ocl}) sketch how this construction can
  be captured by an automated process; its implementation is described 
  elsewhere.}
\endisatagafp
\isatagannexa
\footnote{Our example of \inlineisar+Employee_AnalysisModel+ (see
  \autoref{ex:employee-analysis:uml} and
  \autoref{ex:employee-analysis:ocl}) sketch how this construction can
  be captured by an automated process; its implementation is described 
  elsewhere.}
\endisatagannexa
Moreover, all derived
execution rules are by construction type-safe (which would be an
issue, if we had chosen to use an object universe construction in
Zermelo-Fraenkel set theory as an alternative approach to subtyping.).
In more detail, our theory gives answers and concrete solutions to a
number of open major issues for the UML/OCL standardization:
\begin{enumerate}
\item the role of the two exception elements \inlineisar+invalid+ and
  \inlineisar+null+, the former usually assuming strict evaluation
  while the latter ruled by non-strict evaluation.
\item the functioning of the resulting four-valued logic, together
  with safe rules (for example \inlineisar+foundation9+ --
  \inlineisar+foundation12+ in \autoref{sec:localVal}) that allow a
  reduction to two-valued reasoning as required for many automated
  provers. The resulting logic still enjoys the rules of a strong
  Kleene Logic in the spirit of the Amsterdam
  Manifesto~\cite{cook.ea::amsterdam:2002}.
\item the complicated life resulting from the two necessary
  equalities: the standard's ``strict weak referential equality'' as
  default (written \inlineisar+_ \<doteq> _+ throughout this document) and
  the strong equality (written \inlineisar+_ \<triangleq> _+), which
  follows the logical Leibniz principle that ``equals can be replaced
  by equals.''  Which is not necessarily the case if
  \inlineisar+invalid+ or objects of different states are involved.
\item a type-safe representation of objects and a clarification of the
  old idea of a one-to-one correspondence between object
  representations and object-id's, which became a state invariant.
\item a simple concept of state-framing via the novel operator
  \inlineocl+_->oclIsModifiedOnly()+ and its consequences for strong
  and weak equality.
\item a semantic view on subtyping clarifying the role of static and
  dynamic type (aka \emph{apparent} and \emph{actual} type in Java
  terminology), and its consequences for casts, dynamic type-tests,
  and static types.
\item a semantic view on path expressions, that clarify the role of
  \inlineisar+invalid+ and \inlineisar+null+ as well as the tricky
  issues related to de-referentiation in pre- and post state.
\item an optional extension of the OCL semantics by \emph{infinite}
  sets that provide means to represent ``the set of potential objects
  or values'' to state properties over them (this will be an important
  feature if OCL is intended to become a full-blown code annotation
  language in the spirit of JML~\cite{levens.ea:jml:2007} for semi-automated code verification,
  and has been considered desirable in the Aachen
  Meeting~\cite{brucker.ea:summary-aachen:2013}).
\end{enumerate}
Moreover, we managed to make our theory in large parts executable,
which allowed us to include mechanically checked
\inlineisar+value+-statements that capture numerous corner-cases
relevant for OCL implementors. Among many minor issues, we thus
pin-pointed the behavior of \inlineocl+null+ in collections as well
as in casts and the desired \inlineocl+isKindOf+-semantics of
\inlineocl+allInstances()+.


\section{Lessons Learned}
While our paper and pencil arguments, given
in~\cite{brucker.ea:ocl-null:2009}, turned out to be essentially
correct, there had also been a lesson to be learned: If the logic is
not defined as a Kleene-Logic, having a structure similar to a
complete partial order (CPO), reasoning becomes complicated: several
important algebraic laws break down which makes reasoning in OCL
inherent messy and a semantically clean compilation of OCL formulae to
a two-valued presentation, that is amenable to animators like
KodKod~\cite{torlak.ea:kodkod:2007} or SMT-solvers like
Z3~\cite{moura.ea:z3:2008} completely impractical. Concretely, if the
expression \inlineocl{not(null)} is defined \inlineocl{invalid} (as was
the case in prior versions of the standard~\cite{omg:ocl:2012}), then standard
involution does not hold, \ie, \inlineocl{not(not(A))} = \inlineocl{A}
does not hold universally. Similarly, if \inlineocl{null and null} is
\inlineocl{invalid}, then not even idempotence \inlineocl{X and X} =
\inlineocl{X} holds. We strongly argue in favor of a lattice-like
organization, where \inlineocl{null} represents ``more information''
than \inlineocl{invalid} and the logical operators are monotone with
respect to this semantical ``information ordering.''

A similar experience with prior paper and pencil arguments was our
investigation of the object-oriented data-models, in particular
path-expressions ~\cite{brucker.ea:path-expressions:2013}. The final
presentation is again essentially correct, but the technical details
concerning exception handling lead finally to a continuation-passing
style of the (in future generated) definitions for accessors, casts
and tests.  Apparently, OCL semantics (as many other ``real''
programming and specification languages) is meanwhile too complex to
be treated by informal arguments solely.

Featherweight OCL makes several minor deviations from the standard and
showed how the previous constructions can be made correct and
consistent, and the DNF-normalization as well as $\delta$-closure laws
(necessary for a transition into a two-valued presentation of OCL
specifications ready for interpretation in SMT solvers
(see~\cite{brucker.ea:ocl-testing:2010} for details)) are valid in
Featherweight OCL.

\section{Conclusion and Future Work}
Featherweight OCL concentrates on formalizing the semantics of a core
subset of OCL in general and in particular on formalizing the
consequences of a four-valued logic (\ie, OCL versions that support,
besides the truth values \inlineocl{true} and \inlineocl{false} also
the two exception values \inlineocl{invalid} and \inlineocl{null}).

In the following, we outline the following future extensions to use
Featherweight OCL for a concrete fully fledged tool for OCL.  There are
essentially five extensions necessary:
\begin{itemize}
\item development of a compiler that compiles a textual or CASE
  tool representation (\eg, using XMI or the textual syntax of
  the USE tool~\cite{richters:precise:2002}) of class
  models into an object-oriented data type theory automatically.
\item Full support of OCL standard syntax in a front-end parser;
  Such a parser could also generate the necessary casts as well as
  converting standard OCL to Featherweight OCL as well as providing
  ``normalizations'' such as converting multiplicities of class
  attributes to into OCL class invariants.
\item a setup for translating Featherweight OCL into a two-valued
  representation as described
  in~\cite{brucker.ea:ocl-testing:2010}. As, in real-world scenarios,
  large parts of {UML}/{OCL} specifications are defined (\eg,
  from the default multiplicity \inlineocl{1} of an attributes
  \inlineocl{x}, we can directly infer that for all valid states
  \inlineocl{x} is neither \inlineocl{invalid} nor \inlineocl{null}),
  such a translation enables both an integration of fast constraint solvers
  such as Z3 as well as test-case generation scenarios as described in 
  \cite{brucker.ea:ocl-testing:2010}.
\item a setup in Featherweight OCL of the Nitpick
  animator~\cite{blanchette.ea:nitpick:2010}. It remains to be shown
  that the standard, Kodkod~\cite{torlak.ea:kodkod:2007} based
  animator in Isabelle can give a similar quality of animation as the
  OCLexec Tool~\cite{krieger.ea:generative:2010}
\item a code-generator setup for Featherweight OCL for Isabelle's
  code generator. For example, the Isabelle code generator supports
  the generation of F\#, which would allow to use {OCL}
  specifications for testing arbitrary .net-based applications.
\end{itemize}
The first two extensions are sufficient to provide a formal proof
environment for OCL 2.5 similar to \holocl while the remaining
extensions are geared towards increasing the degree of proof
automation and usability as well as providing a tool-supported test
methodology for {UML}/{OCL}.


Our work shows that developing a machine-checked formal semantics of
recent {OCL} standards still reveals significant
inconsistencies---even though this type of research is not new. In
fact, we started our work already with the 1.x series of {OCL}. The
reasons for this ongoing consistency problems of {OCL} standard are
manifold. For example, the consequences of adding an additional
exception value to OCL 2.2 are widespread across the whole language
and many of them are also quite subtle. Here, a machine-checked formal
semantics is of great value, as one is forced to formalize all details
and subtleties.  Moreover, the standardization process of the {OMG},
in which standards (\eg, the {UML} infrastructure and the {OCL}
standard) that need to be aligned closely are developed quite
independently, are prone to ad-hoc changes that attempt to align these
standards. And, even worse, updating a standard document by voting on
the acceptance (or rejection) of isolated text changes does not help
either. Here, a tool for the editor of the standard that helps to
check the consistency of the whole standard after each and every
modifications can be of great value as well.


%%% Local Variables:
%%% mode: latex
%%% TeX-master: "root"

