\part{A Formal Semantics of OCL 2.3 in Isabelle/HOL}
\section{Formal and Technical Background}
\subsection{Validity and Evaluations}
The topmost goal of the formal semantics is to define the
\emph{validity statement}:
\begin{equation*}
   (\sigma, \sigma') \isasymMathOclValid P\mi{,}
\end{equation*}
where $\sigma$ is the pre-state and $\sigma'$ the post-state of the
underlying system and $P$ is a Boolean expression (a
\emph{formula}). The assertion language of $P$ is composed of
\begin{inparaenum}[1)]
\item operators on built-in data structures such as Boolean or set,
\item operators of the user-defined data-model such as accessors,
  type-casts and tests, and
\item user-defined, side-effect-free methods.
\end{inparaenum}
Informally, a formula $P$ is valid if and only if its evaluation in
the context $(\sigma, \sigma')$ yields true. As all types in \holocl
are extended by the special element $\isasymbottom$ denoting
undefinedness, we define formally:
\begin{equation*}
(\sigma, \sigma') \isasymMathOclValid P \equiv \bigl(P(\sigma, \sigma') =
\lift{\HolTrue}\bigr)\mi{.}
\end{equation*}
Since all operators of the assertion language depend on the context $(\sigma,
\sigma')$ and result in values that can be $\isasymbottom$, all expressions can
be viewed as \emph{evaluations} from $(\sigma, \sigma')$ to a type $\up{\tau}$.
All types of expressions are of a form captured by
 \begin{equation*}
 \V{}{\alpha} \defeq \state{} \times \state{} \to \up{\alpha}  \mi{,}
 \end{equation*}
where $\state{}$ stands for the system state and $\state{} \times
\state{}$ describes the pair of pre-state and post-state and
$\_\defeq\_$ denotes the type abbreviation.

The \OCL semantics~\cite[Annex A]{omg:ocl:2003} uses different interpretation
functions for invariants and pre-conditions; we achieve their semantic effect by
a syntactic transformation $\__\text{pre}$ which replaces all accessor functions
$\getAttrib{\_}{a}$ by their counterparts
$\getAttrib{\_}{a\isasymOclATpre}$. For example, $(\getAttrib{\self}{a} >
5)_\text{pre}$ is just $(\getAttrib{\self}{a\isasymOclATpre} > 5)$.

\subsection{Strict Operations}%
\label{sec:strict-ops}%
\enlargethispage{1em}
An operation is called strict if it returns $\isasymbottom$ if one of its
arguments is $\isasymbottom$. Most \acs{ocl} operations are strict, \eg, the
Boolean negation is formally presented as:
\begin{align*}
I\semantics{\isasymOclNot X}\tau & \equiv
  \begin{cases}
    \lift{\lnot \drop{I\semantics{X} \tau}}
    &\text{if $I\semantics{X}\tau\isasymnoteq\isasymbottom$}, \\
     \isasymbottom & \text{otherwise}\,.
   \end{cases}
\end{align*}
where $\tau = (\sigma, \sigma')$ and $I\semantics{\_}$ is a notation marking the
\holocl constructs to be defined. This notation is motivated by the definitions
in the \acs{ocl} standard~\cite{omg:ocl:2003}. In our case, $I\semantics{\_}$ is
just the identity, \ie, $I\semantics{X}\equiv X$.  These constructs, \ie,
$\isasymOclNot \_$ are \HOL functions (in this case of \HOL type
$\V{}{\HolBoolean} \to \V{}{\HolBoolean}$) that can be viewed as
\emph{transformers on evaluations}.

The binary case of the integer addition is analogous:
\begin{align*}
I\semantics{X \isasymMathOclPlus Y}\ap \tau & \equiv
  \begin{cases}
    \lift{\drop{I\semantics{X}\ap \tau }+\drop{I\semantics{Y}\ap \tau}}
    &\text{if $I\semantics{X}\ap\tau\isasymnoteq\isasymbottom$
    and $I\semantics{Y}\ap\tau\isasymnoteq\isasymbottom$}, \\
     \isasymbottom & \text{otherwise}\mi{.}
   \end{cases}
\end{align*}
Here, the operator $\_+\_$ on the right refers to the integer
\acs{hol} operation with type $[\HolInteger, \HolInteger] \to
\HolInteger$.  The type of the corresponding strict \holocl operator
$\_\isasymMathOclPlus\_$ is $[\V{}{\HolInteger}, \V{}{\HolInteger}]
\to \V{}{\HolInteger}$.
%
A slight variation of this definition scheme is used for
the operators on collection types such as \holocl sets or sequences:
\begin{align*}
I\semantics{X\isasymOclUnion\oP Y\cP} \tau & \equiv
  \begin{cases}
    S\lift{\drop{I\semantics{X}
\tau}\isasymunion\drop{I\semantics{Y} \tau}}
&\text{if $I\semantics{X}\tau\isasymnoteq\isasymbottom$
   and $I\semantics{Y}\tau\isasymnoteq\isasymbottom$,} \\
     \isasymbottom & \text{otherwise.}
   \end{cases}
\end{align*}
Here, $S$ (``smash'') is a function that maps a lifted set $\lift{X}$
to $\bottom$ if and only if $\bottom \in X$ and to the identity
otherwise. Smashedness of collection types is the natural extension of
the strictness principle for data structures.

Intuitively, the type expression $\V{}{\tau}$ is a representation of
the type that corresponds to the \holocl type $\tau$.
We introduce the following type abbreviations:
\begin{align*}
  \OclBoolean[]                  &\defeq \V{}{\HolBoolean}\mi{,}
  &   \OclSet[]{\alpha}      &\defeq \V{}{\HolSet{\alpha}}\mi{,}\\
  \OclInteger[]                  &\defeq \V{}{\HolInteger}\mi{, and}
  & \OclSequence[]{\alpha} &\defeq \V{}{\HolList{\alpha}}\mi{.}
\end{align*}
The mapping of an expression \inlineocl+E+ of
\holocl type \inlineocl+T+ to a \acs{hol} expression $E$ of \acs{hol} type $T$
is injective and preserves well-typedness.

\subsection{Boolean Operators}
There is a small number of explicitly stated exceptions from the general rule
that \holocl operators are strict: the strong equality, the definedness operator
and the logical connectives.  As a prerequisite, we define the logical constants
for truth, absurdity and undefinedness.  We write these definitions as follows:
\begin{align*}
 I\semantics{\OclTrue} \tau \equiv \lift{\HolTrue}\mi{,} & \qquad
 I\semantics{\OclFalse}\tau \equiv \lift{\HolFalse}\mi{, and} & \quad
 I\semantics{\OclUndef}\tau \equiv \bottom\mi{.}
\end{align*}

\holocl has a \emph{strict equality} $\_ \isasymMathOclStrictEq \_$. On the
primitive types, it is defined similarly to the integer addition; the case for
objects is discussed later. For logical purposes, we introduce also a
\emph{strong equality} $\_ \isasymMathOclStrongEq \_$ which is defined as
follows:
\begin{align*}
  I\semantics{X \isasymMathOclStrongEq Y}\ap\tau \equiv  (I\semantics{X}\ap\tau =
  I\semantics{Y}\ap \tau)\mi{,}
\end{align*}
where the $\_ = \_$ operator on the right denotes the logical equality of
\acs{hol}. The undefinedness test is defined by $X\isasymOclIsInvalid \equiv
(X \isasymMathOclStrongEq \OclUndef)$. The strong equality can be
used to state reduction rules like: $\tau \isasymMathOclValid (\OclUndef
\isasymMathOclStrictEq X) \isasymMathOclStrongEq \OclUndef$.  The \acs{ocl}
standard requires a \acl{skl}. In particular:
\begin{align*}
  I\semantics{X \isasymOclAnd~Y}\tau & \equiv
  \begin{cases}
    \lift{\drop{x}\land\drop{y}} &\text{if
      $x\isasymnoteq\isasymbottom$
      and $y \isasymnoteq\isasymbottom$}, \\
    \lift{\HolFalse}&\text{if $x = \lift{\HolFalse}$ or $y
      = \lift{\HolFalse}$}, \\
     \isasymbottom & \text{otherwise}\mi{.}
  \end{cases}
\end{align*}
where $x = I\semantics{X}\tau $ and $y = I\semantics{Y}\tau$.
The other Boolean connectives were
just shortcuts: $X \isasymOclOr Y \equiv \isasymOclNot (\isasymOclNot X
\isasymOclAnd\; \isasymOclNot Y)$ and $X \isasymOclImplies Y \equiv \isasymOclNot
X \isasymOclOr Y$.


\subsection{Object-oriented Data Structures}%
Now we turn to several families of operations that the user implicitly defines
when stating a class model as logical context of a specification. This is the
part of the language where object-oriented features such as type casts, accessor
functions, and tests for dynamic types come into play. Syntactically, a class
model provides a collection of classes ($C_1$, \ldots, $C_n$),
an inheritance relation $\_ < \_$ on
classes and a collection of attributes $A_{C_i}$ associated to classes.  Semantically,
a class model means a collection of accessor functions (denoted $\_\text{.a}::C_i
\rightarrow B$ and $\getAttrib{\_}{a\isasymOclATpre}::C_i \rightarrow B$ for
$\text{a}\in A_{C_i}$ and $B\in \{\V{}{\up{\ldots}}, C_1, \ldots, C_n\}$),
type casts that can change the static type of an object of a class
(denoted $\typeCast{\_}{\text{$C_i$}}$ of type $C_j \rightarrow C_i$) and
two dynamic type tests
(denoted $\isType{\text{$C_i$}}{\_}$ and $\isKind{\text{$C_i$}}{\_}$ ).
A precise formal definition can be found in~\cite{brucker.ea:semantics:2009}.

\subsubsection{Class models: A simplified semantics.}
In this section, we will have to clarify the notions of \emph{object
  identifiers}, \emph{object representations}, \emph{class types} and
\emph{state}. We will give a formal model for this, that will satisfy all
properties discussed in the subsequent section except one
(see~\cite{brucker.ea:extensible:2008-b} for the complete model).

First, object identifiers are captured by an abstract type $\oid$ comprising
countably many elements and a special element $\nullid$.
%
Second, object representations model ``a piece of typed memory,'' \ie, a kind of
record comprising administration information and the information for all
attributes of an object; here, the primitive types as well as collections over
them are stored directly in the object representations, class types and
collections over them are represented by $\oid$'s (respectively lifted
collections over them).
%
Third, the class type $C$ will be the type of such an object representation: $C
\defeq (\oid \times \Tag{C} \times A_1 \times \cdots \times A_k )$ where a
unique tag-type $\Tag{C}$ (ensuring type-safety) is created for each class type,
where the types $A_1$, \ldots, $A_k$ are the attribute types (including
inherited attributes) with class types substituted by $\oid$. The function
$\HolOclOidOf$ projects the first component, the $\oid$, out of an object
representation.
%
Fourth,
for a  class model $M$ with the classes $C_1$, \ldots, $C_n$, we
define states as partial functions from $\oid$'s to object representations
satisfying a \emph{state invariant} $\inv_\sigma$:
\begin{gather*}
  \state{} \defeq \{ f :: \oid \isasymrightharpoonup (C_1 + \ldots +
  C_n) \ap|\ap \inv_\sigma(f) \}
\end{gather*}
where $\inv_\sigma(f)$ states two conditions:
\begin{inparaenum}[1)]
\item there is no object representation for $\nullid$: $\nullid \notin (\dom\ap f)$.
\item there is a``one-to-one'' correspondence between object representations and
  $\oid$'s: $\forall \mathit{oid} \in \dom\ap f\spot oid = \HolOclOidOf\ap
  \drop{f(\mathit{oid})}$.
\end{inparaenum}
The latter condition is also mentioned in~\cite[Annex A]{omg:ocl:2003} and
goes back to Mark Richters~\cite{richters:precise:2002}.

\subsection{The Accessors}
On states built over object universes, we can now define accessors, casts, and
type tests of an object model. We consider the case of an attribute $\mathrm{a}$
of class $C$ which has the simple class type $D$ (not a primitive type, not a
collection):
\begin{gather*}
  I\semantics{\getAttrib[]{\self}{a}} (\sigma, \sigma')   \equiv
  \begin{cases}
\bottom & \text{if $O = \bottom \lor \HolOclOidOf\ap\drop{O}\notin \dom~\sigma'$} \\
    \getType[]{D} \mathit{u}
    & \text{if $\sigma' (\getAttrib[0]{\getType[]{C}\drop{\sigma'(\HolOclOidOf\ap\drop{O})}}{a})
      = \lift{\mathit{u}}$,}\\
\bottom & \text{otherwise.}
\end{cases}
\end{gather*}
\begin{gather*}
I\semantics{\getAttrib[]{\self}{a@pre}} (\sigma, \sigma')   \equiv
  \begin{cases}
   \bottom & \text{if $O = \bottom\lor \HolOclOidOf\ap\drop{O}\notin \dom~\sigma$} \\
    \getType[]{D} \mathit{u}
    & \text{if $\sigma (\getAttrib[]{\getType[]{C}\drop{\sigma(\HolOclOidOf\ap\drop{O})}}{a})
      = \lift{\mathit{u}}$,}\\
\bottom & \text{otherwise.}
  \end{cases}
\end{gather*}
where $O = I\semantics{\self}(\sigma, \sigma')$.  Here, $\getType[]{D}$ is the
projection function from the object universe to $\up{D}$, and
$\getAttrib[]{x}{a}$ is the projection of the attribute from the class type (the
Cartesian product). For simple class types, we have to evaluate expression
$\self$, get an object representation (or undefined), project the attribute,
de-reference it in the pre or post state and project the class object from the
object universe ($\getType[]{D}$ may yield $\bottom$ if the element in the
universe does not correspond to a $D$ object representation.)  In the case for a
primitive type attribute, the de-referentiation step is left out, and in the
case of a collection over class types, the elements of the collection have to be
point-wise de-referenced and smashed.

In our model accessors always yield (type-safe) object representations; not
$\oid$'s. Thus, a dangling reference, \ie, one that is \emph{not} in $\dom~
\sigma$, results in $\isasymOclUndefined$ (this is a subtle
difference to~\cite[Annex A]{omg:ocl:2003} where the undefinedness is detected
one de-referentiation step later). The strict equality $\_
\isasymMathOclStrictEq \_$ must be defined via $\HolOclOidOf$ when applied to
objects. It satisfies $(\isasymOclInvalid \isasymMathOclStrictEq X)
\isasymMathOclStrongEq \isasymOclInvalid$.

The definitions of casts and type tests can be found
in~\cite{brucker.ea:extensible:2008-b}, together with other details of the
construction above and its automation in \holocl.

\section{A Proposal for an OCL 2.1 Semantics}\label{sec:hol-ocl2-semantics}
In this section, we describe our \OCL 2.1 semantics proposal as an increment to
the \OCL 2.0 semantics (underlying \holocl and essentially formalizing
\cite[Annex A]{omg:ocl:2003}). In later versions of the
standard~\cite{omg:ocl:2006} the formal semantics appendix reappears although
being incompatible with the normative parts of the standard.  Not all rules
shown here are formally proven; technically, these are informal proofs ``with a
glance'' on the formal proofs shown in the previous section.

\subsection{Revised Operations on Primitive Types}
In \UML, and since~\cite{omg:ocl:2006} in \OCL, all primitive types comprise the
$\Null{}$-element, modeling the possibility to be non-existent. From a
functional language perspective, this corresponds to the view that each basic
value is a type like \inlinesml{int option} as in \acs{sml}. Technically, this
results in lifting any primitive type twice:
\begin{gather*}
  \OclInteger[]  \defeq \V{}{\up{\HolInteger}}\mi{, etc.}
\end{gather*}
and basic operations have to take the null elements into account. The
distinguishable undefined and null-elements were defined as follows:
\begin{align*}
 I\semantics{\OclUndef}\tau \equiv \bottom  &\mi{ and}  \quad
 I\semantics{\Null{\OclInteger[]}}\tau \equiv \lift{\bottom}\mi{.}
\end{align*}

An interpretation (consistent with~\cite{omg:ocl:2006}) is that
$\Null{\OclInteger[]} + 3 = \isasymOclInvalid$, and due to commutativity, we
postulate $3 + \Null{\OclInteger[]} = \isasymOclInvalid$, too.  The necessary
modification of the semantic interpretation looks as follows:

\begin{align*}
I\semantics{X \isasymMathOclPlus Y}\ap \tau & \equiv
  \begin{cases}
    \lift{\lift{\drop{\drop{x}}+\drop{\drop{y}}}}
    &\text{if $x\isasymnoteq\isasymbottom$, $y\isasymnoteq\isasymbottom$, $\drop{x}\isasymnoteq\isasymbottom$ and $\drop{y}\isasymnoteq\isasymbottom$}\\
     \isasymbottom & \text{otherwise}\mi{.}
   \end{cases}
\end{align*}
where $x = I\semantics{X}\ap \tau$ and $y = I\semantics{Y}\ap \tau$. The
resulting principle here is that operations on the primitive types Boolean,
Integer, Real, and String treat null as invalid (except $\_
\isasymMathOclStrictEq \_ $, $\_\isasymOclIsInvalid$, $\_\isasymOclIsUndefined$,
casts between the different representations of $\isasymOclNull$, and type-tests).
\enlargethispage{1em}

This principle is motivated by our intuition that invalid represents known
errors, and null-arguments of operations for Boolean, Integer, Real, and String
belong to this class. Thus, we must also modify the logical operators such that
$\Null{\OclBoolean[]}~\isasymOclAnd~\isasymOclFalse \isasymMathOclStrongEq
\isasymOclFalse$ and $\Null{\OclBoolean[]}~\isasymOclAnd~\isasymOclTrue
\isasymMathOclStrongEq \isasymMathOclUndefined$.

With respect to definedness reasoning, there is a price to pay. For most basic
operations we have the rule:
\begin{multline*}
  \isasymOclNot (X+Y)\isasymOclIsInvalid \isasymMathOclStrongEq
  (\isasymOclNot X\isasymOclIsUndefined)\\
\isasymOclAnd~
(\isasymOclNot Y\isasymOclIsUndefined)
\end{multline*}
where the test $x\isasymOclIsUndefined$ covers  two cases:
$x\isasymOclIsInvalid$ and $x \isasymMathOclStrictEq \isasymOclNull$ (\ie, $x$
is $\isasymOclInvalid$ or $\isasymOclNull$).  As a consequence, for the inverse
case $(X+Y)\isasymOclIsInvalid$\footnote{The same holds for
  $(X+Y)\isasymOclIsUndefined$.} there are four possible cases for the failure
instead of two in the semantics described in~\cite{omg:ocl:2003}: each
expression can be an erroneous $\isasymOclNull{}$, or report an error. However,
since all built-in \OCL operations yield non-null elements (\eg, we have the rule
$\isasymOclNot (X + Y \isasymMathOclStrictEq \Null{\OclInteger[]})$), a
pre-computation can drastically reduce the number of cases occurring in
expressions except for the base case of variables (\eg, parameters of operations
and $\self$ in invariants). For these cases, it is desirable that implicit
pre-conditions were generated as default, ruling out the $\Null{}$ case. A
convenient place for this are the multiplicities, which can be set to
\inlineocl{1} (\ie, \inlineocl{1..1}) and will be interpreted as being non-null
(see discussion in \autoref{sec:multiplicities} for more details).

Besides, the case for collection types is analogous: in addition to the invalid
collection, there is a $\Null{Set(T)}$ collection as well as collections that
contain null values (such as $\text{Set}\{\Null{T}\}$) but never
$\isasymOclUndefined$.

\subsection{Null in Class Types}
It is a viable option to rule out undefinedness in object-graphs \emph{as such}.
The essential source for such undefinedness are $\oid$'s which do not occur in
the state, \ie, which represent ``dangling references.'' Ruling out
undefinedness as result of object accessors would correspond to a world where an
accessor is always set explicitly to $\Null{}$ or to a defined object; in a
programming language without explicit deletion and where constructors always
initialize their arguments (\eg, \Specsharp~\cite{barnett.ea:spec:2004}), this
may suffice. Semantically, this can be modeled by strengthening the state
invariant $\inv_\sigma$ by adding clauses that state that in each object
representation all $\oid$'s are either $\nullid$ or element of the domain of the
state.

We deliberately decided against this option for the following reasons:
\begin{enumerate}
\item \emph{methodologically} we do not like to constrain the semantics of \OCL
  without clear reason; in particular, ``dangling references'' exist in C and
  \Cpp{} programs and it might be necessary to write contracts for them, and
\item \emph{semantically}, the condition ``no dangling references'' can only be
  formulated with the complete knowledge of all classes and their layout in
  form of object representations.  This restricts the \OCL semantics to a closed
  world model.\footnote{In our presentation, the definition of \inlineocl+state+
    in \autoref{sec:hol-ocl-semantics} assumes a closed world. This limitation
    can be easily overcome by leaving ``polymorphic holes'' in our object
    representation universe, \ie, by extending the type sum in the state
    definition to $C_1 + \cdots + C_n + \alpha$.  The details of the management
    of universe extensions are involved, but implemented in \holocl
    (see~\cite{brucker.ea:extensible:2008-b} for details). However, these
    constructions exclude knowing the set of sub-$\oid$'s in advance.}
\end{enumerate} We can model $\Null{}$-elements as object-representations
with $\nullid$ as their $\oid$:
\begin{definition}[Representation of $\Null{}$-Elements]
  Let $C_i$ be a class type with the attributes $A_1, \ldots, A_n$. Then we
  define its null object representation by:
  \begin{gather*}
    I\semantics{\Null{Ci}}\tau \equiv \lift{(\nullid, \Tag{arb}, a_1, \ldots,
    a_n )}
  \end{gather*}
  where the $a_i$ are $\bottom$ for primitive types and collection
  types, and $\nullid$ for simple class types. $\Tag{arb}$ is an arbitrary
  underspecified constant of the tag-type.
\end{definition}

Due to the outermost lifting, the null object representation is a defined value,
and due to its special reference $\nullid$ and the state invariant, it is a
typed value not ``living'' in the state.  The $\Null{T}$-elements are not equal,
but isomorphic: Each type, has its own unique $\Null{T}$-element; they can be
mapped, \ie, casted, isomorphic to each other. In \holocl, we can overload
constants by parametrized polymorphism which allows us to drop the index in this
environment.

The referential strict equality allows us to write $\self
\isasymMathOclStrictEq \isasymOclNull$ in \OCL. Recall that $\_ \isasymMathOclStrictEq \_$
is based on the projection $\HolOclOidOf$ from object-representations.

\subsection{Revised Accessors}
%Having introduced $\Null{}$-elements,
The modification of the accessor functions
is now straight-forward:
\begin{gather*}
  I\semantics{\getAttrib[]{\obj}{a}} (\sigma, \sigma')   \equiv
  \begin{cases}
\bottom & \text{if $I\semantics{\obj}  (\sigma, \sigma') = \bottom \!\lor\! \HolOclOidOf\drop{ I\semantics{\obj}  (\sigma, \sigma')}\notin \dom\;\sigma'$} \\
\Null{D} & \text{if $\getAttrib[0]{\getType[]{C}\drop{\sigma'(\HolOclOidOf\drop{I\semantics{\obj}  (\sigma, \sigma')})}}{a} = \nullid$} \\
    \getType[]{D} \mathit{u}
    & \text{if $\sigma' (\getAttrib[0]{\getType[]{C}\drop{\sigma'(\HolOclOidOf\drop{ I\semantics{\obj}  (\sigma, \sigma')})}}{a})  = \lift{\mathit{u}}$,}\\
\bottom & \text{otherwise.}
  \end{cases}
\end{gather*}

The definitions for type-cast and dynamic type test---which are not explicitly
shown in this paper, see~\cite{brucker.ea:extensible:2008-b} for details---can
be generalized accordingly. In the sequel, we will discuss the resulting
properties of these modified accessors.

\clearpage
All functions of the induced signature are strict.  This means that this holds
for accessors, casts and tests, too:
\begin{multline*}
  \getAttrib{\isasymOclUndefined}{a} \isasymMathOclStrongEq \isasymOclUndefined
  \qquad
  \typeCast{\isasymOclUndefined}{C} \isasymMathOclStrongEq \isasymOclUndefined
  \\
  \isType{C}{\isasymOclUndefined} \isasymMathOclStrongEq \isasymOclUndefined
\end{multline*}
Casts on $\Null{}$ are always valid, since they have an
individual dynamic type and can be casted to any other null-element due to their
isomorphism.
\begin{multline*}
  \getAttrib{\Null{A}}{a} \isasymMathOclStrongEq \isasymOclUndefined
  \qquad
  \typeCast{\Null{A}}{B} \isasymMathOclStrongEq \Null{B}
  \\%quad
  \isType{A}{\Null{A}} \isasymMathOclStrongEq \isasymOclTrue
\end{multline*}
for all attributes $a$ and classes $A$, $B$, $C$ where $C<B<A$. These rules are
further exceptions from the standard's general rule that $\Null{}$ may never be
passed as first (``$\self$'') argument.

\subsection{Other Operations on States}
Defining $\_\isasymOclAllInstances$
is straight-forward; the only difference is the property
$T\isasymOclAllInstances\isasymOclExcludes(\isasymOclNull)$ which is a
consequence of the fact that $\Null{}$'s are values and do not ``live'' in the
state.  In our semantics which admits states with ``dangling references,'' it is
possible to define a counterpart to \inlineocl+_.oclIsNew()+ called
\inlineocl+_.oclIsDeleted()+ which asks if an object id (represented by an object
representation) is contained in the pre-state, but not the post-state.

\OCL does not guarantee that an operation only modifies the path-expressions
mentioned in the postcondition, \ie, it allows arbitrary relations from
pre-states to post-states.  This framing problem is well-known (one of the
suggested solutions is~\cite{kosiuczenko:specification:2006}). We define
\begin{ocl}
 (S:Set(OclAny))->oclIsModifiedOnly():Boolean
\end{ocl}
where \inlineocl|S| is a set of object representations, encoding
a set of $\oid$'s. The semantics of this operator is defined such that
for any object whose $\oid$ is \emph{not }represented in \inlineocl|S|
and that is defined in pre and post state, the corresponding object representation will not change
in the state transition:
\begin{gather*}
I\semantics{X\isasymMathOclIsModifiedOnly} (\sigma, \sigma')  \equiv
  \begin{cases}
    \isasymbottom & \text{if $X' = \bottom$}  \\
     \lift{\isasymforall i \isasymin M\spot
        \sigma~i = \sigma'~i} & \text{otherwise}\mi{.}
   \end{cases}
\end{gather*}
where $X' = I\semantics{X} (\sigma, \sigma')$ and $M=
(\dom~\sigma\cap\dom~\sigma') - \{ \HolOclOidOf x |~x \in\drop{X'}\}$.  Thus, if
we require in a postcondition \inlineocl|Set{}->oclIsModifiedOnly()| and exclude via
\inlineocl+_.oclIsNew()+ and \inlineocl+_.oclIsDeleted()+ the existence of new
or deleted objects, the operation is a query in the sense of the \OCL standard, \ie,
the \inlineocl|isQuery| property is true. So, whenever we have $ \tau
\isasymMathOclValid X\isasymMathOclIsModifiedOnly$ and $ \tau
\isasymMathOclValid X\isasymOclExcludes(s.a)$, we can infer that $\tau
\isasymMathOclValid s.a = s.a\isasymOclATpre$ (if they are valid).

\section{Attribute Values}\label{sec:multiplicities}
Depending on the specified multiplicity, the evaluation of an attribute can
yield a value or a collection of values. A multiplicity defines a lower bound as
well as a possibly infinite upper bound on the cardinality of the attribute's
values.

\subsection{Single-Valued Attributes}\label{subsec:single-valued-properties}
If the upper bound specified by the attribute's multiplicity is one, then an
evaluation of the attribute yields a single value. Thus, the evaluation result
is not a collection. If the lower bound specified by the multiplicity is zero,
the evaluation is not required to yield a non-null value. In this case an
evaluation of the attribute can return $\isasymOclNull$ to indicate an
absence of value.

To facilitate accessing attributes with multiplicity \inlineocl{0..1}, the \OCL
standard states that single values can be used as sets by calling collection
operations on them. This implicit conversion of a value to a
\inlineocl{Set} is not defined by the standard. We argue that the resulting set
cannot be constructed the same way as when evaluating a \inlineocl{Set}
literal. Otherwise, $\isasymOclNull$ would be mapped to the singleton set
containing $\isasymOclNull$, but the standard demands that
the resulting set is empty in this case. The conversion should instead
be defined as follows:
\begin{ocl}
context OclAny::asSet():T
  post: if self $\isasymMathOclStrictEq$ null then result $\isasymMathOclStrictEq$ Set{}
        else result $\isasymMathOclStrictEq$ Set{self} endif
\end{ocl}
% Changed self.isTypeOf(OCLVoid) to self = null to make it easier for the superficial reader

\subsection{Collection-Valued Attributes}\label{subsec:collection-valued-properties}
If the upper bound specified by the attribute's multiplicity is larger than one,
then an evaluation of the attribute yields a collection of values.  This raises
the question whether $\isasymOclNull$ can belong to this collection. The \OCL
standard states that $\isasymOclNull$ can be owned by collections. However, if
an attribute can evaluate to a collection containing $\isasymOclNull$, it is not
clear how multiplicity constraints should be interpreted for this attribute. The
question arises whether the $\isasymOclNull$ element should be counted or not
when determining the cardinality of the collection. Recall that $\isasymOclNull$
denotes the absence of value in the case of a cardinality upper bound of one, so
we would assume that $\isasymOclNull$ is not counted. On the other hand, the
operation \inlineocl{size} defined for collections in \OCL does count
$\isasymOclNull$.

We propose to resolve this dilemma by regarding multiplicities as optional. This
point of view complies with the \UML standard, that does not require lower and
upper bounds to be defined for multiplicities.\footnote{We are however aware
  that a well-formedness rule of the \UML standard does define a default bound
  of one in case a lower or upper bound is not specified.} In case a
multiplicity is specified for an attribute, \ie, a lower and an upper bound
are provided, we require any collection the attribute evaluates to to not
contain $\isasymOclNull$. This allows for a straightforward interpretation of
the multiplicity constraint. If bounds are not provided for an attribute, we
consider the attribute values to not be restricted in any way. Because in
particular the cardinality of the attribute's values is not bounded, the result
of an evaluation of the attribute is of collection type. As the range of values
that the attribute can assume is not restricted, the attribute can evaluate to a
collection containing $\isasymOclNull$. The attribute can also evaluate to
$\isasymOclInvalid$. Allowing multiplicities to be optional in this way gives
the modeler the freedom to define attributes that can assume the full ranges of
values provided by their types. However, we do not permit the omission of
multiplicities for association ends, since the values of association ends are
not only restricted by multiplicities, but also by other constraints enforcing
the semantics of associations. Hence, the values of association ends cannot be
completely unrestricted.

\subsection{The Precise Meaning of Multiplicity Constraints}
We are now ready to define the meaning of multiplicity constraints by giving
equivalent invariants written in \OCL. Let \inlineocl{a} be an attribute of a
class \inlineocl{C} with a multiplicity specifying a lower bound $m$ and an
upper bound $n$. Then we can define the multiplicity constraint on the values of
attribute \inlineocl{a} to be equivalent to the following invariants written in
\OCL:
\begin{ocl}
context C inv lowerBound: a->size() >= m
          inv upperBound: a->size() <= n
          inv notNull: not a->includes(null)
\end{ocl}
If the upper bound $n$ is infinite, the second invariant is omitted. For the
definition of these invariants we are making use of the conversion of single
values to sets described in \autoref{subsec:single-valued-properties}. If $n
\leq 1$, the attribute \inlineocl{a} evaluates to a single value, which is then
converted to a \inlineocl{Set} on which the \inlineocl{size} operation is
called.

If a value of the attribute \inlineocl{a} includes a reference to a non-existent
object, the attribute call evaluates to $\isasymOclInvalid$. As a result, the
entire expressions evaluate to $\isasymOclInvalid$, and the invariants are not
satisfied. Thus, references to non-existent objects are ruled out by these
invariants. We believe that this result is appropriate, since we argue that the
presence of such references in a system state is usually not intended and likely
to be the result of an error. If the modeler wishes to allow references to
non-existent objects, she can make use of the possibility described above to
omit the multiplicity.



\input{session}
%%% Local Variables: 
%%% mode: latex
%%% TeX-master: "root"

